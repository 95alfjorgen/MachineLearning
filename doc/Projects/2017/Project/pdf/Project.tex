%%
%% Automatically generated file from DocOnce source
%% (https://github.com/hplgit/doconce/)
%%
%%


%-------------------- begin preamble ----------------------

\documentclass[%
oneside,                 % oneside: electronic viewing, twoside: printing
final,                   % draft: marks overfull hboxes, figures with paths
10pt]{article}

\listfiles               %  print all files needed to compile this document

\usepackage{relsize,makeidx,color,setspace,amsmath,amsfonts,amssymb}
\usepackage[table]{xcolor}
\usepackage{bm,ltablex,microtype}

\usepackage[pdftex]{graphicx}

\usepackage[T1]{fontenc}
%\usepackage[latin1]{inputenc}
\usepackage{ucs}
\usepackage[utf8x]{inputenc}

\usepackage{lmodern}         % Latin Modern fonts derived from Computer Modern

% Hyperlinks in PDF:
\definecolor{linkcolor}{rgb}{0,0,0.4}
\usepackage{hyperref}
\hypersetup{
    breaklinks=true,
    colorlinks=true,
    linkcolor=linkcolor,
    urlcolor=linkcolor,
    citecolor=black,
    filecolor=black,
    %filecolor=blue,
    pdfmenubar=true,
    pdftoolbar=true,
    bookmarksdepth=3   % Uncomment (and tweak) for PDF bookmarks with more levels than the TOC
    }
%\hyperbaseurl{}   % hyperlinks are relative to this root

\setcounter{tocdepth}{2}  % levels in table of contents

% --- fancyhdr package for fancy headers ---
\usepackage{fancyhdr}
\fancyhf{} % sets both header and footer to nothing
\renewcommand{\headrulewidth}{0pt}
\fancyfoot[LE,RO]{\thepage}
% Ensure copyright on titlepage (article style) and chapter pages (book style)
\fancypagestyle{plain}{
  \fancyhf{}
  \fancyfoot[C]{{\footnotesize \copyright\ 1999-2017, "Data Analysis and Machine Learning FYS-MAT3155/FYS4155":"http://www.uio.no/studier/emner/matnat/fys/FYS3155/index-eng.html". Released under CC Attribution-NonCommercial 4.0 license}}
%  \renewcommand{\footrulewidth}{0mm}
  \renewcommand{\headrulewidth}{0mm}
}
% Ensure copyright on titlepages with \thispagestyle{empty}
\fancypagestyle{empty}{
  \fancyhf{}
  \fancyfoot[C]{{\footnotesize \copyright\ 1999-2017, "Data Analysis and Machine Learning FYS-MAT3155/FYS4155":"http://www.uio.no/studier/emner/matnat/fys/FYS3155/index-eng.html". Released under CC Attribution-NonCommercial 4.0 license}}
  \renewcommand{\footrulewidth}{0mm}
  \renewcommand{\headrulewidth}{0mm}
}

\pagestyle{fancy}


% prevent orhpans and widows
\clubpenalty = 10000
\widowpenalty = 10000

% --- end of standard preamble for documents ---


% insert custom LaTeX commands...

\raggedbottom
\makeindex
\usepackage[totoc]{idxlayout}   % for index in the toc
\usepackage[nottoc]{tocbibind}  % for references/bibliography in the toc

%-------------------- end preamble ----------------------

\begin{document}

% matching end for #ifdef PREAMBLE

\newcommand{\exercisesection}[1]{\subsection*{#1}}


% ------------------- main content ----------------------



% ----------------- title -------------------------

\thispagestyle{empty}

\begin{center}
{\LARGE\bf
\begin{spacing}{1.25}
Project on Machine Learning
\end{spacing}
}
\end{center}

% ----------------- author(s) -------------------------

\begin{center}
{\bf \href{{http://www.uio.no/studier/emner/matnat/fys/FYS3155/index-eng.html}}{Data Analysis and Machine Learning FYS-MAT3155/FYS4155}}
\end{center}

    \begin{center}
% List of all institutions:
\centerline{{\small Department of Physics, University of Oslo, Norway}}
\end{center}
    
% ----------------- end author(s) -------------------------

% --- begin date ---
\begin{center}
Fall semester 2017
\end{center}
% --- end date ---

\vspace{1cm}


\subsection*{Using results from Monte Carlo models for machine learning}

\paragraph{Introduction.}
The aim of this project is to use an already developed Monte Carlo program (either the ising Model or a variational Monte Carlo code) to produce, in case of the \href{{https://github.com/CompPhysics/MachineLearning/tree/master/doc/Programs/IsingModel}}{Ising model}, the energy as function of temperature. For the \href{{https://github.com/CompPhysics/MachineLearning/tree/master/doc/Programs/VMC2Electrons/}}{variational Monte Carlo calculation} of interacting electrons in an oscilaltor trap, the data are represented by the ground state energies as functions of the variational parameters.

In its simplest form
the energy of the Ising model is expressed as, without an externally applied magnetic field, 
\[
E=-J\sum_{< kl >}^{N}s_ks_l 
\]
with
$s_k=\pm 1$. The quantity $N$ represents the total number of spins and $J$ is a coupling
constant expressing the strength of the interaction between
neighboring spins.  The symbol $<kl>$ indicates that we sum over
nearest neighbors only. We will assume that we have a ferromagnetic
ordering, viz $J> 0$.  We will use periodic boundary conditions and
the Metropolis algorithm only. Alternatively we can use the supplied variational Monte Carlo program which solves Schroedinger's equation for two interacting electrons in a harmonic oscillator trap. Both codes can be found at the webpage of the \href{{https://github.com/CompPhysics/MachineLearning/tree/master/doc/Programs/}}{course under programs}. 

\paragraph{Part a): Producing the data.}
If we opt for the Ising model code, we need to generate for every span over the lattice and output of the total energy and the magnetic moment (if we wish to study this quantity as well), that is compute and write to file as function of temperature
$\langle E\rangle$ and $\langle \vert M\vert \rangle$.
We will use a fixed lattice siste of $L\times L = 40 \times 40$. Make sure the calculations have been equilibrated and compute these expectation values for 
for $T\in [1.0,3.0]$ with a step in
temperature $\Delta T=0.1$ or smaller.  This are the data you will use to estimate the standard deviation in the next part of the project.
You should keep a file for the different temperatures as you will need to compute the standard deviation for each temperature. 


Alternatively you can run the variational Monte Carlo program for two interacting electrons confined to move in a harmonic oscillator trap. Here you will produce the variational expectaction value of the energy as a function of the variational parameters $\alpha$ and $\beta$. You could single out only a set of such parameters after you have found the minimum energy. 

These two cases form then our training data which we will use  for our estimates of the standard deviation.


\paragraph{Part b): Estimating the standard deviation.}
This part deals with widely used resampling methods to find the standard deviation on our data sets. The methods we can use are the Blocking method, the Bootstrap method and the Jackknife method. The latter two are examples of so-called resampling methods. \href{{https://github.com/CompPhysics/MachineLearning/tree/master/doc/Programs/Sampling}}{The functions provided under the program folder} encode all three methods. The typical situation is to generate data using a C++ or Fortan code, as in part a) above. These data are then used by a post-analysis program in order to perform a statistical analysis. 

In case of the Ising model, the final product will be a table of the expectation values of the energy (or the magnetic moment) as function of temperature and with a proper standard deviation. These data will then enter our analysis in parts c-e). We will call these data the training data. 

If you opt for the variational Monte Carlo program, you need to make a table of the ground state energy as function of various variational parameters (you can limit yourself to $\alpha$ only) with proper standard deviation estimations.

Your task here is thus to use the data from part a) and generate the standard deviation using one (or more) of the methods discussed above. In general, the Bootstrap method is the most widely used one. 

These data will then be used in our next step, where we will try to obtain a functional approximation to our data sets. 

\paragraph{Part c): Fitting the data using regression analysis and other methods.}
More text to come
\paragraph{Part d): Introducing Bayesian statistics.}
More text to come

\paragraph{Part e): Studying the Ising model or the VMC results with Neural networks.}
More text to come

\subsection*{Background literature}

If you wish to read more about the Ising model and statistical physics here are three suggestions.

\begin{itemize}
  \item \href{{http://www.worldscientific.com/worldscibooks/10.1142/5660}}{M. Plischke and B. Bergersen}, \emph{Equilibrium Statistical Physics}, World Scientific, see chapters 5 and 6.

  \item \href{{http://www.cambridge.org/no/academic/subjects/physics/computational-science-and-modelling/guide-monte-carlo-simulations-statistical-physics-4th-edition?format=HB}}{D. P. Landau and K. Binder}, \emph{A Guide to Monte Carlo Simulations in Statistical Physics}, Cambridge, see chapters 2,3 and 4.

  \item \href{{https://global.oup.com/academic/product/monte-carlo-methods-in-statistical-physics-9780198517979?cc=no&lang=en&}}{M. E. J. Newman and T. Barkema}, \emph{Monte Carlo Methods in Statistical Physics}, Oxford, see chapters 3 and 4.
\end{itemize}

\noindent
\subsection*{Introduction to numerical projects}

Here follows a brief recipe and recommendation on how to write a report for each
project.

\begin{itemize}
  \item Give a short description of the nature of the problem and the eventual  numerical methods you have used.

  \item Describe the algorithm you have used and/or developed. Here you may find it convenient to use pseudocoding. In many cases you can describe the algorithm in the program itself.

  \item Include the source code of your program. Comment your program properly.

  \item If possible, try to find analytic solutions, or known limits in order to test your program when developing the code.

  \item Include your results either in figure form or in a table. Remember to        label your results. All tables and figures should have relevant captions        and labels on the axes.

  \item Try to evaluate the reliabilty and numerical stability/precision of your results. If possible, include a qualitative and/or quantitative discussion of the numerical stability, eventual loss of precision etc.

  \item Try to give an interpretation of you results in your answers to  the problems.

  \item Critique: if possible include your comments and reflections about the  exercise, whether you felt you learnt something, ideas for improvements and  other thoughts you've made when solving the exercise. We wish to keep this course at the interactive level and your comments can help us improve it.

  \item Try to establish a practice where you log your work at the  computerlab. You may find such a logbook very handy at later stages in your work, especially when you don't properly remember  what a previous test version  of your program did. Here you could also record  the time spent on solving the exercise, various algorithms you may have tested or other topics which you feel worthy of mentioning.
\end{itemize}

\noindent

% ------------------- end of main content ---------------

\end{document}

