%%
%% Automatically generated file from DocOnce source
%% (https://github.com/hplgit/doconce/)
%%
%%


%-------------------- begin preamble ----------------------

\documentclass[%
oneside,                 % oneside: electronic viewing, twoside: printing
final,                   % draft: marks overfull hboxes, figures with paths
10pt]{article}

\listfiles               %  print all files needed to compile this document

\usepackage{relsize,makeidx,color,setspace,amsmath,amsfonts,amssymb}
\usepackage[table]{xcolor}
\usepackage{bm,ltablex,microtype}

\usepackage[pdftex]{graphicx}

\usepackage[T1]{fontenc}
%\usepackage[latin1]{inputenc}
\usepackage{ucs}
\usepackage[utf8x]{inputenc}

\usepackage{lmodern}         % Latin Modern fonts derived from Computer Modern

% Hyperlinks in PDF:
\definecolor{linkcolor}{rgb}{0,0,0.4}
\usepackage{hyperref}
\hypersetup{
    breaklinks=true,
    colorlinks=true,
    linkcolor=linkcolor,
    urlcolor=linkcolor,
    citecolor=black,
    filecolor=black,
    %filecolor=blue,
    pdfmenubar=true,
    pdftoolbar=true,
    bookmarksdepth=3   % Uncomment (and tweak) for PDF bookmarks with more levels than the TOC
    }
%\hyperbaseurl{}   % hyperlinks are relative to this root

\setcounter{tocdepth}{2}  % levels in table of contents

% --- fancyhdr package for fancy headers ---
\usepackage{fancyhdr}
\fancyhf{} % sets both header and footer to nothing
\renewcommand{\headrulewidth}{0pt}
\fancyfoot[LE,RO]{\thepage}
% Ensure copyright on titlepage (article style) and chapter pages (book style)
\fancypagestyle{plain}{
  \fancyhf{}
  \fancyfoot[C]{{\footnotesize \copyright\ 1999-2018, "Data Analysis and Machine Learning FYS-STK3155/FYS4155":"http://www.uio.no/studier/emner/matnat/fys/FYS3155/index-eng.html". Released under CC Attribution-NonCommercial 4.0 license}}
%  \renewcommand{\footrulewidth}{0mm}
  \renewcommand{\headrulewidth}{0mm}
}
% Ensure copyright on titlepages with \thispagestyle{empty}
\fancypagestyle{empty}{
  \fancyhf{}
  \fancyfoot[C]{{\footnotesize \copyright\ 1999-2018, "Data Analysis and Machine Learning FYS-STK3155/FYS4155":"http://www.uio.no/studier/emner/matnat/fys/FYS3155/index-eng.html". Released under CC Attribution-NonCommercial 4.0 license}}
  \renewcommand{\footrulewidth}{0mm}
  \renewcommand{\headrulewidth}{0mm}
}

\pagestyle{fancy}


% prevent orhpans and widows
\clubpenalty = 10000
\widowpenalty = 10000

% --- end of standard preamble for documents ---


% insert custom LaTeX commands...

\raggedbottom
\makeindex
\usepackage[totoc]{idxlayout}   % for index in the toc
\usepackage[nottoc]{tocbibind}  % for references/bibliography in the toc

%-------------------- end preamble ----------------------

\begin{document}

% matching end for #ifdef PREAMBLE

\newcommand{\exercisesection}[1]{\subsection*{#1}}


% ------------------- main content ----------------------



% ----------------- title -------------------------

\thispagestyle{empty}

\begin{center}
{\LARGE\bf
\begin{spacing}{1.25}
Project on Machine Learning
\end{spacing}
}
\end{center}

% ----------------- author(s) -------------------------

\begin{center}
{\bf \href{{http://www.uio.no/studier/emner/matnat/fys/FYS3155/index-eng.html}}{Data Analysis and Machine Learning FYS-STK3155/FYS4155}}
\end{center}

    \begin{center}
% List of all institutions:
\centerline{{\small Department of Physics, University of Oslo, Norway}}
\end{center}
    
% ----------------- end author(s) -------------------------

% --- begin date ---
\begin{center}
May 2018
\end{center}
% --- end date ---

\vspace{1cm}


\subsection*{Machine learning (ML) approaches to data from Ising model calculations}

\paragraph{Introduction.}
The aim of this project is to use an already developed Monte Carlo program for the one-dimensional and two-dimensional \href{{https://github.com/CompPhysics/MachineLearning/tree/master/doc/Programs/IsingModel}}{Ising model}, in order to produce the spin configurations for a series of energies $E_i$ (10000 in total) for a system of $L=40$ spins in one dimension and $L=40\times 40$ in two dimensions at three different temperatures.
In its simplest form the energy of the Ising model is expressed as, without an externally applied magnetic field, 
\[
E=-J\sum_{< kl >}^{N}s_ks_l 
\]
with
$s_k=\pm 1$. The quantity $N$ represents the total number of spins and $J$ is a coupling
constant expressing the strength of the interaction between
neighboring spins.  The symbol $<kl>$ indicates that we sum over
nearest neighbors only. We will assume that we have a ferromagnetic
ordering, viz $J> 0$.  We will use periodic boundary conditions and
the Metropolis algorithm only. The spins take values $-1$ and $+1$ only. 


We will use the Ising model to generate our training data and will focus mainly on supervised training. We will follow closely the recent article of \href{{https://arxiv.org/abs/1803.08823}}{Mehta et al, arXiv 1803.08823}. This article stands out as an excellent review on machine learning (ML) algorithms applied to typical physics problems. The added benefit is that each figure and model presented in \href{{https://physics.bu.edu/~pankajm/MLnotebooks.html}}{this article is accompanied by its jupyter notebook}. This means that we can start using these and compare with our own results. In case you wish to use their data for the Ising model, their data can be downloaded from the same link which lists to the jupyter notebooks. See also at the end of the project description for more information on how to install various Python packages. 



With the abovementioned  configurations we will determine, using first various
regression methods, the value of the coupling constant for the energy
of the one-dimensional Ising model. Thereafter, we will use the
two-dimensional data, but now computed at different temperatures, in
order to classify the phase of the Ising model. Below the critical
temperature, the system will be in a so-called ferromagnetic
phase. Close to the critical temperature, the final magnetization becomes smaller and smaller in absolute value
 while above the critical temperature,
the net magnetization is zero.  This classification  case, that is the
two-dimensional Ising model, will be studied using logistic regression, a \textbf{random forest}
algorithm and deep neural networks.

You should try to program at least one of these methods yourself (choose the one you prefer). 
Feel free to use the notebooks to benchmark your code.  If you wish to write your own C++ or Fortran program for say a simple neural network model, please feel free to do so.
You can then benchmark your results against the above jupyter notebooks.  More information can also be found at the link for the lecture notes of \href{{https://compphysics.github.io/MachineLearning/doc/web/course.html}}{FYS-STK4155}.


We recommend that you form groups of 2-3 students and try to
collaborate on the notebooks, develop your own software and discuss
the final presentations. You can collaborate on all these topics. The
final presentation should include an overview of popular machine
learning algorithms as introduction and motivation. Thereafter you
discuss the explicit Ising model data and how you have implemented the
ML algorithms discussed here, discuss their pros and cons and try to
develop your own code for at least one of these algorithms.  You are
encouraged to use the abovementioned notebooks as starting point and
guidance.  The duration of your presentation should at most be 30
mins. Allow for approximately 15 mins for discussions and questions.



\paragraph{Part a): Producing the data.}
You can use the Ising model data from the article of Mehta \emph{et al.}, or generate your own data. 
If you opt for using your own Ising model code, you need to generate $10000$ energy configurations with their spin orientations after the system has reached its most likely state. These energies and their corresponding spin orientations
represent then your data. 
We will use a fixed lattice of $L\times L = 40 \times 40$ spins in two dimensions and $L=40$ spins in one dimension. 
Make sure the calculations have been equilibrated. For the two-dimensional system, compute the configurations 
for three values of the temperature, namely $T=0.75$ (ordered phase), $T=2.3$ (near the critical point)  and $T=4.0$ (disordered phase).
For the one-dimensional system it suffices to compute the various configurations for one temperature only, say $T=2.0$.
These are the data you will use to study different ML algorithms.
We generate our data with $J=1$.

\paragraph{Part b): Estimating the coupling constant of the one-dimensional Ising model.}
We start with the one-dimensional Ising model and use the data we have generated with $J=1$. Use linear regression, Lasso and Ridge regression as described section 6 and in Notebook 4 of \href{{https://physics.bu.edu/~pankajm/ML-Notebooks/HTML/NB_CVI-linreg_ising.html}}{Mehta *et al.*}. Discuss the methods and how they perform in computing the coupling constant $J$. Give a critical analysis and discuss how to evaluate the \emph{cost function}. You should feel free to write your own code, see also
the lecture notes of \href{{https://compphysics.github.io/MachineLearning/doc/web/course.html}}{FYS-STK4155}, in particular te material on least square methods.  You can use scikit-learn to perform these analyses. See below for instruction on how to install scikit-learn.

\paragraph{Part c): Determine the phase of the two-dimensional Ising model.}
We switch now to binary classification methods and use logistic regression to define the phases of the Ising model.
Use described section 7 and in Notebook 6 of \href{{https://physics.bu.edu/~pankajm/ML-Notebooks/HTML/NB_CVII-logreg_ising.html}}{Mehta *et al.*}. Discuss the methods and how they perform. Give a critical analysis and discuss how to evaluate the \emph{cost function}. You should feel free to write your own code. Use thereafter the \emph{random forests} algorithm to classify the same phases as done with logistic regression and discuss the pros and cons of these methods. For \emph{random forests} you can use \href{{https://physics.bu.edu/~pankajm/ML-Notebooks/HTML/NB_CVIII-randomforests_ising.html}}{notebook 9} of Mehta \emph{et al.}

You can use scikit-learn to perform these analyses. See below for instruction on how to install scikit-learn.

\paragraph{Part d): Classifying the Ising model phase using neural networks.}
We end the classification problem of the phases of the Ising model by employing the algorithm for so-called feed-forward deep neural networks (see section 9 of Mehta \emph{et al.}). The method is described in \href{{https://physics.bu.edu/~pankajm/ML-Notebooks/HTML/NB_CIX-DNN_ising_TFlow.html}}{notebook 12}. 

You can use tensorflow  to perform these analyses. See below for instruction on how to install tensorflow.


\paragraph{Part e):  Summary.}
You should make a summary of the various methods and their pros and cons. For the final presentation, you should have at least coded one of these methods yourself and discussed the evaluation of the cost function. 

\subsection*{Background literature}

On Machine Learning we recommend strongly the article of Mehta \emph{et al.}
Textbooks on Machine Learning can be found at the \href{{https://github.com/CompPhysics/MachineLearning/tree/master/doc/Textbooks}}{Github address of FYS-STK4155}, see in particular Marsland's text. 

\begin{itemize}
  \item \href{{https://arxiv.org/abs/1803.08823}}{Mehta et al, arXiv 1803.08823}, \emph{A high-bias, low-variance introduction to Machine Learning for physicists}, ArXiv:1803.08823.
\end{itemize}

\noindent
If you wish to read more about the Ising model and statistical physics here are three suggestions.

\begin{itemize}
  \item \href{{http://www.worldscientific.com/worldscibooks/10.1142/5660}}{M. Plischke and B. Bergersen}, \emph{Equilibrium Statistical Physics}, World Scientific, see chapters 5 and 6.

  \item \href{{http://www.cambridge.org/no/academic/subjects/physics/computational-science-and-modelling/guide-monte-carlo-simulations-statistical-physics-4th-edition?format=HB}}{D. P. Landau and K. Binder}, \emph{A Guide to Monte Carlo Simulations in Statistical Physics}, Cambridge, see chapters 2,3 and 4.

  \item \href{{https://global.oup.com/academic/product/monte-carlo-methods-in-statistical-physics-9780198517979?cc=no&lang=en&}}{M. E. J. Newman and T. Barkema}, \emph{Monte Carlo Methods in Statistical Physics}, Oxford, see chapters 3 and 4.
\end{itemize}

\noindent
\subsection*{Introduction to numerical projects}

Here follows a brief recipe and recommendation on how to write a report for each
project.

\begin{itemize}
  \item Give a short description of the nature of the problem and the eventual  numerical methods you have used.

  \item Describe the algorithm you have used and/or developed. Here you may find it convenient to use pseudocoding. In many cases you can describe the algorithm in the program itself.

  \item Include the source code of your program. Comment your program properly.

  \item If possible, try to find analytic solutions, or known limits in order to test your program when developing the code.

  \item Include your results either in figure form or in a table. Remember to        label your results. All tables and figures should have relevant captions        and labels on the axes.

  \item Try to evaluate the reliabilty and numerical stability/precision of your results. If possible, include a qualitative and/or quantitative discussion of the numerical stability, eventual loss of precision etc.

  \item Try to give an interpretation of you results in your answers to  the problems.

  \item Critique: if possible include your comments and reflections about the  exercise, whether you felt you learnt something, ideas for improvements and  other thoughts you've made when solving the exercise. We wish to keep this course at the interactive level and your comments can help us improve it.

  \item Try to establish a practice where you log your work at the  computerlab. You may find such a logbook very handy at later stages in your work, especially when you don't properly remember  what a previous test version  of your program did. Here you could also record  the time spent on solving the exercise, various algorithms you may have tested or other topics which you feel worthy of mentioning.
\end{itemize}

\noindent
\subsection*{Software and needed installations}

If you have Python installed (we recommend Python3) and you feel pretty familiar with installing different packages, 
we recommend that you install the following Python packages via \textbf{pip} as
\begin{enumerate}
\item pip install numpy scipy matplotlib ipython scikit-learn tensorflow sympy pandas pillow
\end{enumerate}

\noindent
For Python3, replace \textbf{pip} with \textbf{pip3}.

See below for a discussion of \textbf{tensorflow} and \textbf{scikit-learn}. 

For OSX users we recommend also, after having installed Xcode, to install \textbf{brew}. Brew allows 
for a seamless installation of additional software via for example
\begin{enumerate}
\item brew install python3
\end{enumerate}

\noindent
For Linux users, with its variety of distributions like for example the widely popular Ubuntu distribution
you can use \textbf{pip} as well and simply install Python as 
\begin{enumerate}
\item sudo apt-get install python3  (or python for python2.7)
\end{enumerate}

\noindent
etc etc. 

If you don't want to install various Python packages with their dependencies separately, we recommend two widely used distrubutions which set up  all relevant dependencies for Python, namely
\begin{enumerate}
\item \href{{https://docs.anaconda.com/}}{Anaconda} Anaconda is an open source distribution of the Python and R programming languages for large-scale data processing, predictive analytics, and scientific computing, that aims to simplify package management and deployment. Package versions are managed by the package management system \textbf{conda}

\item \href{{https://www.enthought.com/product/canopy/}}{Enthought canopy}  is a Python distribution for scientific and analytic computing distribution and analysis environment, available for free and under a commercial license.
\end{enumerate}

\noindent
Popular software packages written in Python for ML are

\begin{itemize}
\item \href{{http://scikit-learn.org/stable/}}{Scikit-learn}, 

\item \href{{https://www.tensorflow.org/}}{Tensorflow},

\item \href{{http://pytorch.org/}}{PyTorch} and 

\item \href{{https://keras.io/}}{Keras}.
\end{itemize}

\noindent
These are all freely available at their respective GitHub sites. They 
encompass communities of developers in the thousands or more. And the number
of code developers and contributors keeps increasing.


% ------------------- end of main content ---------------

\end{document}

