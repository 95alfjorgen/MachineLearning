%%
%% Automatically generated file from DocOnce source
%% (https://github.com/hplgit/doconce/)
%%
%%
% #ifdef PTEX2TEX_EXPLANATION
%%
%% The file follows the ptex2tex extended LaTeX format, see
%% ptex2tex: http://code.google.com/p/ptex2tex/
%%
%% Run
%%      ptex2tex myfile
%% or
%%      doconce ptex2tex myfile
%%
%% to turn myfile.p.tex into an ordinary LaTeX file myfile.tex.
%% (The ptex2tex program: http://code.google.com/p/ptex2tex)
%% Many preprocess options can be added to ptex2tex or doconce ptex2tex
%%
%%      ptex2tex -DMINTED myfile
%%      doconce ptex2tex myfile envir=minted
%%
%% ptex2tex will typeset code environments according to a global or local
%% .ptex2tex.cfg configure file. doconce ptex2tex will typeset code
%% according to options on the command line (just type doconce ptex2tex to
%% see examples). If doconce ptex2tex has envir=minted, it enables the
%% minted style without needing -DMINTED.
% #endif

% #define PREAMBLE

% #ifdef PREAMBLE
%-------------------- begin preamble ----------------------

\documentclass[%
oneside,                 % oneside: electronic viewing, twoside: printing
final,                   % draft: marks overfull hboxes, figures with paths
10pt]{article}

\listfiles               %  print all files needed to compile this document

\usepackage{relsize,makeidx,color,setspace,amsmath,amsfonts,amssymb}
\usepackage[table]{xcolor}
\usepackage{bm,ltablex,microtype}

\usepackage[pdftex]{graphicx}

\usepackage{ptex2tex}
% #ifdef MINTED
\usepackage{minted}
\usemintedstyle{default}
% #endif

\usepackage[T1]{fontenc}
%\usepackage[latin1]{inputenc}
\usepackage{ucs}
\usepackage[utf8x]{inputenc}

\usepackage{lmodern}         % Latin Modern fonts derived from Computer Modern

% Hyperlinks in PDF:
\definecolor{linkcolor}{rgb}{0,0,0.4}
\usepackage{hyperref}
\hypersetup{
    breaklinks=true,
    colorlinks=true,
    linkcolor=linkcolor,
    urlcolor=linkcolor,
    citecolor=black,
    filecolor=black,
    %filecolor=blue,
    pdfmenubar=true,
    pdftoolbar=true,
    bookmarksdepth=3   % Uncomment (and tweak) for PDF bookmarks with more levels than the TOC
    }
%\hyperbaseurl{}   % hyperlinks are relative to this root

\setcounter{tocdepth}{2}  % levels in table of contents

% --- fancyhdr package for fancy headers ---
\usepackage{fancyhdr}
\fancyhf{} % sets both header and footer to nothing
\renewcommand{\headrulewidth}{0pt}
\fancyfoot[LE,RO]{\thepage}
% Ensure copyright on titlepage (article style) and chapter pages (book style)
\fancypagestyle{plain}{
  \fancyhf{}
  \fancyfoot[C]{{\footnotesize \copyright\ 1999-2018, "Data Analysis and Machine Learning FYS-STK3155/FYS4155":"http://www.uio.no/studier/emner/matnat/fys/FYS3155/index-eng.html". Released under CC Attribution-NonCommercial 4.0 license}}
%  \renewcommand{\footrulewidth}{0mm}
  \renewcommand{\headrulewidth}{0mm}
}
% Ensure copyright on titlepages with \thispagestyle{empty}
\fancypagestyle{empty}{
  \fancyhf{}
  \fancyfoot[C]{{\footnotesize \copyright\ 1999-2018, "Data Analysis and Machine Learning FYS-STK3155/FYS4155":"http://www.uio.no/studier/emner/matnat/fys/FYS3155/index-eng.html". Released under CC Attribution-NonCommercial 4.0 license}}
  \renewcommand{\footrulewidth}{0mm}
  \renewcommand{\headrulewidth}{0mm}
}

\pagestyle{fancy}


% prevent orhpans and widows
\clubpenalty = 10000
\widowpenalty = 10000

% --- end of standard preamble for documents ---


% insert custom LaTeX commands...

\raggedbottom
\makeindex
\usepackage[totoc]{idxlayout}   % for index in the toc
\usepackage[nottoc]{tocbibind}  % for references/bibliography in the toc

%-------------------- end preamble ----------------------

\begin{document}

% matching end for #ifdef PREAMBLE
% #endif

\newcommand{\exercisesection}[1]{\subsection*{#1}}


% ------------------- main content ----------------------



% ----------------- title -------------------------

\thispagestyle{empty}

\begin{center}
{\LARGE\bf
\begin{spacing}{1.25}
Project 2 on Machine Learning, deadline November 12
\end{spacing}
}
\end{center}

% ----------------- author(s) -------------------------

\begin{center}
{\bf \href{{http://www.uio.no/studier/emner/matnat/fys/FYS3155/index-eng.html}}{Data Analysis and Machine Learning FYS-STK3155/FYS4155}}
\end{center}

    \begin{center}
% List of all institutions:
\centerline{{\small Department of Physics, University of Oslo, Norway}}
\end{center}
    
% ----------------- end author(s) -------------------------

% --- begin date ---
\begin{center}
Nov 2, 2018
\end{center}
% --- end date ---

\vspace{1cm}


\subsection{Classification and Regression, from linear and logistic regression to neural networks}

The main aim of this project is to study both classification and
regression problems, starting with the regression algorithms studied
in project 1. We will include logistic regresion for classification
problems and write our own multilayer perceptron code for studying
both regression and classification problems.  The codes developed in
project 1, including bootstrap and/or cross-validation as well as the
computation of the mean-squared error and the R2 score function can
also be utilized (and included in logistic regression and the neural
network codes) in the present analysis.

We will use the so-called Ising model for  our training data and will
focus on supervised training. We will follow closely the recent
article of \href{{https://arxiv.org/abs/1803.08823}}{Mehta et al, arXiv
1803.08823}. This article stands
out as an excellent review on machine learning (ML) algorithms.
The added benefit is that each figure and
model presented in \href{{https://physics.bu.edu/~pankajm/MLnotebooks.html}}{this article is accompanied by its jupyter
notebook}. This
means that we can start using these and compare with our own results.
They provide also the data set for the regression and classification
analysis that we will explore.  In this sense, with their available
notebooks, it makes life easier since we can compare our own codes
with their codes.




With the abovementioned configurations we will determine, using first
various regression methods, the value of the coupling constant for the
energy of the one-dimensional Ising model. Thereafter, we will use the
two-dimensional data, but now computed at different temperatures, in
order to classify the phase of the Ising model. Below the critical
temperature, the system will be in a so-called ferromagnetic
phase. Close to the critical temperature, the final magnetization
becomes smaller and smaller in absolute value while above the critical
temperature, the net magnetization is zero.  This classification case,
that is the two-dimensional Ising model, will be studied using
logistic regression and deep neural networks.  The aim is to develop
your own logistic regression code for the classification of the phases
(this is a binary model) and your multilayer perceptron code for the
classification and regression case. You can compare your own results with those obtained
using \textbf{scikit-learn} or \textbf{tensorflow} or other Python packages such as \textbf{keras} or other.


Feel free to use the notebooks to benchmark your code.  If you wish to
write your own C++ or Fortran program for say a multilayer neural network
model and a logistic regression model, please feel free to do so.  


\paragraph{Part a): Producing the data for the one-dimensional Ising model.}
The model we will employ in our studies is the so-called \href{{https://en.wikipedia.org/wiki/Ising_model}}{Ising
model}.  Together with
models like the \href{{https://en.wikipedia.org/wiki/Potts_model}}{Potts
model} and similar
so-called lattice models, the Ising model has been widely studied in
mathematics (in statistics in particular), physics, \href{{https://journals.aps.org/pre/abstract/10.1103/PhysRevE.93.062402}}{life
science},
chemistry and even in the \href{{https://www.springer.com/gp/book/9781461420316}}{social sciences in order to model social
behavior}. It is a
simple binary value system where the variables of the model (spins often in
physics) can take two values only, for example $\pm 1$ or $0$ and $1$.
The system exhibits a phase transition in two or higher dimensions and
the first person to find the analytical expressions for various
expectation values was the Norwegian chemist \href{{https://en.wikipedia.org/wiki/Lars_Onsager}}{Lars
Onsager} (Nobel prize in
chemistry) after a tour de force mathematics exercise.

In our discussions here we will stay with a physicist's approach and
call the variables for spin. You could replace this with any other
type of binary variables, ranging from a two political parties to blue
and red spheres.  In its simplest form we define the energy of the
system as

\begin{equation*}
  E=-J\sum_{<kl>}^{N}s_ks_l,
\end{equation*}
with  $s_k=\pm 1$, $N$ is the total number of spins,
$J$ is a coupling constant expressing the strength of the interaction
between neighboring spins.

The symbol $<kl>$ indicates that we sum over nearest
neighbors only.
Notice that for $J>0$ it is energetically favorable for neighboring spins
to be aligned. This feature leads to, at low enough temperatures,
a cooperative phenomenon called spontaneous magnetization. That is,
through interactions between nearest neighbors, a given magnetic
moment can influence the alignment of spins  that are separated
from the given spin by a macroscopic distance. These long range correlations
between spins are associated with a long-range order in which
the lattice has a net magnetization in the absence of a magnetic field.



We start by considering the one-dimensional Ising model with nearest neighbor interactions. This model does not exhibit any phase transition.

Consider the 1D Ising model with nearest-neighbor interactions 

\begin{equation*}
  E[\hat{s}]=-J\sum_{j=1}^{N}s_{j}s_{j+1},
\end{equation*}

on a chain of length $N$ with so-called periodic boundary conditions and $S_j=\pm 1$ Ising spin variables.
In one dimension, this model has no phase transition at finite temperature.

In the Python code below we generate, with a coupling coefficient set to $J=1$, a large number of spin configurations say $10000$ as shown in the code below.
It means that our data will be a set of $i=1\ldots n$ points of the form
$\{(E[\boldsymbol{s}^i],\boldsymbol{s}^i)\}$.
Our task is to find the value of $J$ from the data set using linear regression.

Here is the Python code you need to generate the training data, see
also the \href{{https://physics.bu.edu/~pankajm/ML-Notebooks/HTML/NB_CVI-linreg_ising.html}}{notebook of Mehta et
al}.

\bpycod
import numpy as np
import scipy.sparse as sp
np.random.seed(12)

import warnings
#Comment this to turn on warnings
warnings.filterwarnings('ignore')

### define Ising model aprams
# system size
L=40

# create 10000 random Ising states
states=np.random.choice([-1, 1], size=(10000,L))

def ising_energies(states,L):
    """
    This function calculates the energies of the states in the nn Ising Hamiltonian
    """
    J=np.zeros((L,L),)
    for i in range(L):
        J[i,(i+1)%L]-=1.0
    # compute energies
    E = np.einsum('...i,ij,...j->...',states,J,states)

    return E
# calculate Ising energies
energies=ising_energies(states,L)
\epycod

We can now recast the problem as a linear regression model using our codes from project 1.
The way we are going to build our model mimicks the way we could think of finding say the gravitional constant for the graviational force between two planets.
In the absence of any prior knowledge, one sensible choice is the all-to-all Ising model

\[
E_\mathrm{model}[\boldsymbol{s}^i] = - \sum_{j=1}^N \sum_{k=1}^N J_{j,k}s_{j}^is_{k}^i.
\]

Here $i$ represents a particular spin configuration (one of the possible $n$ configurations we generated with the code above).

This model is uniquely defined by the non-local coupling strengths $J_{jk}$ which we want to learn.
The model is linear in $\mathbf{J}$ which makes it possible to use linear regression.

To apply linear regression, we recast this model in the form
\[
E_\mathrm{model}^i \equiv \mathbf{X}^i \cdot \mathbf{J},
\]

where the vectors $\mathbf{X}^i$ represent all two-body interactions
$\{s_{j}^is_{k}^i \}_{j,k=1}^N$, and the index $i$ runs over the
samples in the data set.  To make the analogy complete, we can also
represent the dot product by a single index $p = \{j,k\}$,
i.e.~$\mathbf{X}^i \cdot \mathbf{J}=X^i_pJ_p$.  Note that the
regression model does not include the minus sign, so we expect to
learn negative $J$'s.

With these preliminaries, we are now ready to  reutilize our codes from project 1. 


\paragraph{Part b): Estimating the coupling constant of the one-dimensional Ising model using linear regression.}
We start with the one-dimensional Ising model and use the data we have
generated with $J=1$ in the previous point. Use linear regression,
Lasso and Ridge regression as done in project 1. You can compare your
results with those of \href{{https://physics.bu.edu/~pankajm/ML-Notebooks/HTML/NB_CVI-linreg_ising.html}}{Mehta
et al.}.
Make sure it is the 1D data which is used.

Discuss the methods and how they perform in computing the coupling
constant $J$ and include a bias-variance analysis using either
cross-validation or bootstrap. Discuss also the mean squared error and
the $R2$ score as measures to assess your model.

Give a critical analysis of your results.


\paragraph{Part c): Determine the phase of the two-dimensional Ising model.}
We switch now to binary classification methods and use logistic
regression to define the phases of the Ising model. This means that we switch to the two-dimensional Ising model
and use the data sets generated by \href{{https://physics.bu.edu/~pankajm/ML-Review-Datasets/isingMC/}}{Mehta et al}
These energies and their corresponding spin orientation configurations 
represent then your data.  We will use a fixed lattice of $L\times L =
40 \times 40$ spins in two dimensions. The link above contains data for several temperatures.
The theoretical critical temperature for a phase transition is $T_C\approx 2.269$ in units of energy.
However, for a finite lattice the results representing the critical temperature are slightly higher ($T_C \approx 2.3$). 

Our goal here, using logistic regression, is to train our model to
predict the phase of a sample given the spin configuration, whether it
represents a state above the critical temperature or below.  The
configurations representing states below the critical temperature are
called ordered states (the spins tend to point in one direction,
resulting in a net magnetic moment) while those above the critical
temperature are called disordered. Since a finite lattice like this
does not exhibit a clear sign of a phase transition we will mainly
stay with either orderer or disordered phases. You could include the
critical phase if you want.


Your aim here is thus to read in these data (use the examples from
\href{{https://physics.bu.edu/~pankajm/ML-Notebooks/HTML/NB_CVII-logreg_ising.html}}{Mehta et
al})
and write your own code for doing logistic regression, see the lecture
notes on \href{{https://compphysics.github.io/MachineLearning/doc/pub/LogReg/html/LogReg-bs.html}}{logistic
regression}.


In this case, to evaluate the model, we will use the so-called accuracy score 
instead of the bootstrap or cross-validation as done in the standard linear regression part discussed in b).  Examples of how to define the accuracy score can be found under the neural network slides, see for example the \href{{https://compphysics.github.io/MachineLearning/doc/pub/NeuralNet/html/._NeuralNet-bs047.html}}{slides here}.

To measure the performance of our network we evaluate how well it does
it data it has never seen before, i.e.~the test data.  We measure the
performance of the network using the \emph{accuracy} score.  The accuracy
is as you would expect just the number of images correctly labeled
divided by the total number of images. A perfect classifier will have
an accuracy score of $1$.

\[ 
\text{Accuracy} = \frac{\sum_{i=1}^n I(t_i = y_i)}{n} ,
\]

where $I$ is the indicator function, $1$ if $t_i = y_i$ and $0$
otherwise, where $t_i$ represents the target and $y_i$ the outputs.



In order to find the optimal parameters of your logistic regressor you should
include a gradient descent solver, as discussed in the \href{{https://compphysics.github.io/MachineLearning/doc/pub/Splines/html/Splines-bs.html}}{gradient
descent
lectures}.
Since we don't have so many data points, you may just code the
standard gradient descent with a given learning rate, or even attempt
to use the Newton-Raphson method.  Alternatively, it may be useful for
the next part on neural networks to implement a stochastic gradient
descent with and without mini-batches. Stochastic gradient with mini-batches may give the best results. You could finally compare your code with the output from \textbf{scikit-learn}'s toolbox for
optimization methods applied to logistic regression.


The notebook of \href{{https://physics.bu.edu/~pankajm/ML-Notebooks/HTML/NB_CVII-logreg_ising.html}}{Mehta et al} is highly recommended in order to benchmark your code and results.

\paragraph{Part d): Regression analysis of the one-dimensional Ising model using neural networks.}
Your aim now, and this is the central part of this project, is to
write to your own multilayer perceptron model implementing the back
propagation algorithm discussed in the \href{{https://compphysics.github.io/MachineLearning/doc/pub/NeuralNet/html/NeuralNet-bs.html}}{lecture
slides}. We
start with the regression case discussed in parts a) and b) but train
now the network to find the optimal weights and biases. You are free
to use the codes in the above lecture slides as starting points.

Train your network and compare the results with those from your linear regression code. 
You can test your results against a similar code using _scikit_learn_ (see the examples in the above lecture notes) or \textbf{tensorflow/keras}. 


A useful reference on the back progagation algorithm is \href{{http://neuralnetworksanddeeplearning.com/}}{Nielsen's book}. It is an excellent read.

\paragraph{Part e): Classifying the Ising model phase using neural networks.}
Finally, change now your cost function to the $log$ cross-entropy classification cost function for the case discussed in part c). Train your network again and
compare the results with those from your logistic  regression code i c). 
Here again you can compare your results with those of \href{{https://physics.bu.edu/~pankajm/ML-Notebooks/HTML/NB_CIX-DNN_ising_TFlow.html}}{Mehta et al}. There they used \textbf{tensorflow} to classify the phases. 



\paragraph{Part f) Critical evaluation of the various algorithms.}
After all these glorious calculations, you should now summarize the various algorithms and come with a critical evaluation of their pros and cons. Which algorithm works best for the regression case and which is best for the classification case. These codes will also be part of your final project 3, but now applied to other data sets.




\subsection{Background literature}

\begin{enumerate}
\item The text of Michael Nielsen is highly recommended, see \href{{http://neuralnetworksanddeeplearning.com/}}{Nielsen's book}. It is an excellent read.

\item The textbook of \href{{https://www.springer.com/gp/book/9780387848570}}{Trevor Hastie, Robert Tibshirani, Jerome H. Friedman, The Elements of Statistical Learning, Springer}, chapters 3 and 7 are the most relevant ones for the analysis here. 

\item \href{{https://arxiv.org/abs/1803.08823}}{Mehta et al, arXiv 1803.08823}, \emph{A high-bias, low-variance introduction to Machine Learning for physicists}, ArXiv:1803.08823.
\end{enumerate}

\noindent
If you wish to read more about the Ising model and statistical physics here are three suggestions.

\begin{enumerate}
\item \href{{http://www.worldscientific.com/worldscibooks/10.1142/5660}}{M. Plischke and B. Bergersen}, \emph{Equilibrium Statistical Physics}, World Scientific, see chapters 5 and 6.

\item \href{{http://www.cambridge.org/no/academic/subjects/physics/computational-science-and-modelling/guide-monte-carlo-simulations-statistical-physics-4th-edition?format=HB}}{D. P. Landau and K. Binder}, \emph{A Guide to Monte Carlo Simulations in Statistical Physics}, Cambridge, see chapters 2,3 and 4.

\item \href{{https://global.oup.com/academic/product/monte-carlo-methods-in-statistical-physics-9780198517979?cc=no&lang=en&}}{M. E. J. Newman and T. Barkema}, \emph{Monte Carlo Methods in Statistical Physics}, Oxford, see chapters 3 and 4.
\end{enumerate}

\noindent
\subsection{Introduction to numerical projects}

Here follows a brief recipe and recommendation on how to write a report for each
project.

\begin{itemize}
  \item Give a short description of the nature of the problem and the eventual  numerical methods you have used.

  \item Describe the algorithm you have used and/or developed. Here you may find it convenient to use pseudocoding. In many cases you can describe the algorithm in the program itself.

  \item Include the source code of your program. Comment your program properly.

  \item If possible, try to find analytic solutions, or known limits in order to test your program when developing the code.

  \item Include your results either in figure form or in a table. Remember to        label your results. All tables and figures should have relevant captions        and labels on the axes.

  \item Try to evaluate the reliabilty and numerical stability/precision of your results. If possible, include a qualitative and/or quantitative discussion of the numerical stability, eventual loss of precision etc.

  \item Try to give an interpretation of you results in your answers to  the problems.

  \item Critique: if possible include your comments and reflections about the  exercise, whether you felt you learnt something, ideas for improvements and  other thoughts you've made when solving the exercise. We wish to keep this course at the interactive level and your comments can help us improve it.

  \item Try to establish a practice where you log your work at the  computerlab. You may find such a logbook very handy at later stages in your work, especially when you don't properly remember  what a previous test version  of your program did. Here you could also record  the time spent on solving the exercise, various algorithms you may have tested or other topics which you feel worthy of mentioning.
\end{itemize}

\noindent
\subsection{Format for electronic delivery of report and programs}

The preferred format for the report is a PDF file. You can also use DOC or postscript formats or as an ipython notebook file.  As programming language we prefer that you choose between C/C++, Fortran2008 or Python. The following prescription should be followed when preparing the report:

\begin{itemize}
  \item Use Devilry to hand in your projects, log in  at  \href{{http://devilry.ifi.uio.no}}{\nolinkurl{http://devilry.ifi.uio.no}} with your normal UiO username and password and choose either 'fysstk3155' or 'fysstk4155'. There you can load up the files within the deadline.

  \item Upload \textbf{only} the report file!  For the source code file(s) you have developed please provide us with your link to your github domain.  The report file should include all of your discussions and a list of the codes you have developed.  Do not include library files which are available at the course homepage, unless you have made specific changes to them.

  \item In your git repository, please include a folder which contains selected results. These can be in the form of output from your code for a selected set of runs and input parameters.

  \item In this and all later projects, you should include tests (for example unit tests) of your code(s).

  \item Comments  from us on your projects, approval or not, corrections to be made  etc can be found under your Devilry domain and are only visible to you and the teachers of the course.
\end{itemize}

\noindent
Finally, 
we encourage you to collaborate. Optimal working groups consist of 
2-3 students. You can then hand in a common report. 



\subsection{Software and needed installations}

If you have Python installed (we recommend Python3) and you feel pretty familiar with installing different packages, 
we recommend that you install the following Python packages via \textbf{pip} as
\begin{enumerate}
\item pip install numpy scipy matplotlib ipython scikit-learn tensorflow sympy pandas pillow
\end{enumerate}

\noindent
For Python3, replace \textbf{pip} with \textbf{pip3}.

See below for a discussion of \textbf{tensorflow} and \textbf{scikit-learn}. 

For OSX users we recommend also, after having installed Xcode, to install \textbf{brew}. Brew allows 
for a seamless installation of additional software via for example
\begin{enumerate}
\item brew install python3
\end{enumerate}

\noindent
For Linux users, with its variety of distributions like for example the widely popular Ubuntu distribution
you can use \textbf{pip} as well and simply install Python as 
\begin{enumerate}
\item sudo apt-get install python3  (or python for python2.7)
\end{enumerate}

\noindent
etc etc. 

If you don't want to install various Python packages with their dependencies separately, we recommend two widely used distrubutions which set up  all relevant dependencies for Python, namely
\begin{enumerate}
\item \href{{https://docs.anaconda.com/}}{Anaconda} Anaconda is an open source distribution of the Python and R programming languages for large-scale data processing, predictive analytics, and scientific computing, that aims to simplify package management and deployment. Package versions are managed by the package management system \textbf{conda}

\item \href{{https://www.enthought.com/product/canopy/}}{Enthought canopy}  is a Python distribution for scientific and analytic computing distribution and analysis environment, available for free and under a commercial license.
\end{enumerate}

\noindent
Popular software packages written in Python for ML are

\begin{itemize}
\item \href{{http://scikit-learn.org/stable/}}{Scikit-learn}, 

\item \href{{https://www.tensorflow.org/}}{Tensorflow},

\item \href{{http://pytorch.org/}}{PyTorch} and 

\item \href{{https://keras.io/}}{Keras}.
\end{itemize}

\noindent
These are all freely available at their respective GitHub sites. They 
encompass communities of developers in the thousands or more. And the number
of code developers and contributors keeps increasing.













% ------------------- end of main content ---------------

% #ifdef PREAMBLE
\end{document}
% #endif

