%%
%% Automatically generated file from DocOnce source
%% (https://github.com/hplgit/doconce/)
%%
%%
% #ifdef PTEX2TEX_EXPLANATION
%%
%% The file follows the ptex2tex extended LaTeX format, see
%% ptex2tex: http://code.google.com/p/ptex2tex/
%%
%% Run
%%      ptex2tex myfile
%% or
%%      doconce ptex2tex myfile
%%
%% to turn myfile.p.tex into an ordinary LaTeX file myfile.tex.
%% (The ptex2tex program: http://code.google.com/p/ptex2tex)
%% Many preprocess options can be added to ptex2tex or doconce ptex2tex
%%
%%      ptex2tex -DMINTED myfile
%%      doconce ptex2tex myfile envir=minted
%%
%% ptex2tex will typeset code environments according to a global or local
%% .ptex2tex.cfg configure file. doconce ptex2tex will typeset code
%% according to options on the command line (just type doconce ptex2tex to
%% see examples). If doconce ptex2tex has envir=minted, it enables the
%% minted style without needing -DMINTED.
% #endif

% #define PREAMBLE

% #ifdef PREAMBLE
%-------------------- begin preamble ----------------------

\documentclass[%
oneside,                 % oneside: electronic viewing, twoside: printing
final,                   % draft: marks overfull hboxes, figures with paths
10pt]{article}

\listfiles               %  print all files needed to compile this document

\usepackage{relsize,makeidx,color,setspace,amsmath,amsfonts,amssymb}
\usepackage[table]{xcolor}
\usepackage{bm,ltablex,microtype}

\usepackage[pdftex]{graphicx}

\usepackage{ptex2tex}
% #ifdef MINTED
\usepackage{minted}
\usemintedstyle{default}
% #endif

\usepackage[T1]{fontenc}
%\usepackage[latin1]{inputenc}
\usepackage{ucs}
\usepackage[utf8x]{inputenc}

\usepackage{lmodern}         % Latin Modern fonts derived from Computer Modern

% Hyperlinks in PDF:
\definecolor{linkcolor}{rgb}{0,0,0.4}
\usepackage{hyperref}
\hypersetup{
    breaklinks=true,
    colorlinks=true,
    linkcolor=linkcolor,
    urlcolor=linkcolor,
    citecolor=black,
    filecolor=black,
    %filecolor=blue,
    pdfmenubar=true,
    pdftoolbar=true,
    bookmarksdepth=3   % Uncomment (and tweak) for PDF bookmarks with more levels than the TOC
    }
%\hyperbaseurl{}   % hyperlinks are relative to this root

\setcounter{tocdepth}{2}  % levels in table of contents

% --- fancyhdr package for fancy headers ---
\usepackage{fancyhdr}
\fancyhf{} % sets both header and footer to nothing
\renewcommand{\headrulewidth}{0pt}
\fancyfoot[LE,RO]{\thepage}
% Ensure copyright on titlepage (article style) and chapter pages (book style)
\fancypagestyle{plain}{
  \fancyhf{}
  \fancyfoot[C]{{\footnotesize \copyright\ 1999-2018, "Data Analysis and Machine Learning FYS-STK3155/FYS4155":"http://www.uio.no/studier/emner/matnat/fys/FYS3155/index-eng.html". Released under CC Attribution-NonCommercial 4.0 license}}
%  \renewcommand{\footrulewidth}{0mm}
  \renewcommand{\headrulewidth}{0mm}
}
% Ensure copyright on titlepages with \thispagestyle{empty}
\fancypagestyle{empty}{
  \fancyhf{}
  \fancyfoot[C]{{\footnotesize \copyright\ 1999-2018, "Data Analysis and Machine Learning FYS-STK3155/FYS4155":"http://www.uio.no/studier/emner/matnat/fys/FYS3155/index-eng.html". Released under CC Attribution-NonCommercial 4.0 license}}
  \renewcommand{\footrulewidth}{0mm}
  \renewcommand{\headrulewidth}{0mm}
}

\pagestyle{fancy}


% prevent orhpans and widows
\clubpenalty = 10000
\widowpenalty = 10000

% --- end of standard preamble for documents ---


% insert custom LaTeX commands...

\raggedbottom
\makeindex
\usepackage[totoc]{idxlayout}   % for index in the toc
\usepackage[nottoc]{tocbibind}  % for references/bibliography in the toc

%-------------------- end preamble ----------------------

\begin{document}

% matching end for #ifdef PREAMBLE
% #endif

\newcommand{\exercisesection}[1]{\subsection*{#1}}


% ------------------- main content ----------------------



% ----------------- title -------------------------

\thispagestyle{empty}

\begin{center}
{\LARGE\bf
\begin{spacing}{1.25}
Homework 2
\end{spacing}
}
\end{center}

% ----------------- author(s) -------------------------

\begin{center}
{\bf \href{{http://www.uio.no/studier/emner/matnat/fys/FYS3155/index-eng.html}}{Data Analysis and Machine Learning FYS-STK3155/FYS4155}}
\end{center}

    \begin{center}
% List of all institutions:
\centerline{{\small Department of Physics, University of Oslo, Norway}}
\end{center}
    
% ----------------- end author(s) -------------------------

% --- begin date ---
\begin{center}
Sep 5, 2018
\end{center}
% --- end date ---

\vspace{1cm}


\subsection{Exercise 4}

This exercise is a continuation of exercise 2 from homework 1. We will
use the same function to generate our data set, still staying with a
simple function $y(x)$ which we want to fit using linear regression,
but now extending the analysis to include the Ridge and the Lasso
regression methods. You can use the code under the Regression as an example on how to use the Ridge and the Lasso methods, see the \href{{https://compphysics.github.io/MachineLearning/doc/pub/Regression/html/Regression-bs.html}}{regression slides}). 

We will thus again generate our own dataset for a function $y(x)$ where 
$x \in [0,1]$ and defined by random numbers computed with the uniform
distribution. The function $y$ is a quadratic polynomial in $x$ with
added stochastic noise according to the normal distribution $\cal{N}(0,1)$.

The following simple Python instructions define our $x$ and $y$ values (with 100 data points).
\bpycod
x = np.random.rand(100,1)
y = 5*x*x+0.1*np.random.randn(100,1)
\epycod

\begin{enumerate}
\item Write your own code for the Ridge method (see chapter 3.4 of Hastie \emph{et al.}, equations (3.43) and (3.44)) and compute the parametrization for different values of $\lambda$. Compare and analyze your results with those from exercise 2. Study the dependence on $\lambda$ while also varying the strength of the noise in your expression for $y(x)$. 

\item Repeat the above but using the functionality of \textbf{scikit-learn}. Compare your code with the results from \textbf{scikit-learn}. Remember to run with the same random numbers for generating $x$ and $y$. 

\item Our next step is to study the variance of the parameters $\beta_1$ and $\beta_2$ (assuming that we are parametrizing our function with a second-order polynomial. We will use standard linear regression and the Ridge regression.  You can now opt for either writing your own function that calculates the variance of these paramaters (recall that this is equal to the diagonal elements of the matrix $(\hat{X}^T\hat{X})+\lambda\hat{I})^{-1}$) or use the functionality of \textbf{scikit-learn} and compute their variances. Discuss the results of these variances as functions of $\lambda$. In particular, try to link your discussion with the discussion in Hastie \emph{et al.} and their figure 3.11.

\item Repeat the previous step but add now the Lasso method, see equation (3.53) of Hastie \emph{et al.}. Discuss your results and compare with standard regression and the Ridge regression results. You can write your own code or use the functionality of \textbf{scikit-learn}.  

\item Finally, using \textbf{scikit-learn} or your own code, compute also the mean square error, a risk metric corresponding to the expected value of the squared (quadratic) error defined as
\end{enumerate}

\noindent
\[ MSE(\hat{y},\hat{\tilde{y}}) = \frac{1}{n}
\sum_{i=0}^{n-1}(y_i-\tilde{y}_i)^2, 
\] 
and the $R^2$ score function.
If $\tilde{\hat{y}}_i$ is the predicted value of the $i-th$ sample and $y_i$ is the corresponding true value, then the score $R^2$ is defined as
\[
R^2(\hat{y}, \tilde{\hat{y}}) = 1 - \frac{\sum_{i=0}^{n - 1} (y_i - \tilde{y}_i)^2}{\sum_{i=0}^{n - 1} (y_i - \bar{y})^2},
\]
where we have defined the mean value  of $\hat{y}$ as
\[
\bar{y} =  \frac{1}{n} \sum_{i=0}^{n - 1} y_i.
\]
Discuss these quantities as functions of the variable $\lambda$ in the Ridge and Lasso regression methods. 

\subsection{Exercise 5}

Using the singular value decomposition, show that the variance of the direction vector 
$\hat{z}_i=\hat{X}\hat{v}_i=\hat{u}_1d_1$   is equal to (equation (3.49) of Hastie \emph{et al.})
\[
\mathrm{Var}(\hat{z}_i)=\frac{d_i^2}{N},
\]
where $d_i$ are the singular values of the matrix $\hat{X}$. In Hastie \emph{et al}, the matrix elements of $X$ are centered. The consequence is that the mean values of for example $\hat{u}_i$ are zero. 

Give an interpretation of these results, in particular in connection with the variance of the coefficients you obtained in the previous exercise.   

% ------------------- end of main content ---------------

% #ifdef PREAMBLE
\end{document}
% #endif

