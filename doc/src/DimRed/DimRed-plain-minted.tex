%%
%% Automatically generated file from DocOnce source
%% (https://github.com/hplgit/doconce/)
%%
%%


%-------------------- begin preamble ----------------------

\documentclass[%
oneside,                 % oneside: electronic viewing, twoside: printing
final,                   % draft: marks overfull hboxes, figures with paths
10pt]{article}

\listfiles               %  print all files needed to compile this document

\usepackage{relsize,makeidx,color,setspace,amsmath,amsfonts,amssymb}
\usepackage[table]{xcolor}
\usepackage{bm,ltablex,microtype}

\usepackage[pdftex]{graphicx}

\usepackage{fancyvrb} % packages needed for verbatim environments
\usepackage{minted}
\usemintedstyle{default}

\usepackage[T1]{fontenc}
%\usepackage[latin1]{inputenc}
\usepackage{ucs}
\usepackage[utf8x]{inputenc}

\usepackage{lmodern}         % Latin Modern fonts derived from Computer Modern

% Hyperlinks in PDF:
\definecolor{linkcolor}{rgb}{0,0,0.4}
\usepackage{hyperref}
\hypersetup{
    breaklinks=true,
    colorlinks=true,
    linkcolor=linkcolor,
    urlcolor=linkcolor,
    citecolor=black,
    filecolor=black,
    %filecolor=blue,
    pdfmenubar=true,
    pdftoolbar=true,
    bookmarksdepth=3   % Uncomment (and tweak) for PDF bookmarks with more levels than the TOC
    }
%\hyperbaseurl{}   % hyperlinks are relative to this root

\setcounter{tocdepth}{2}  % levels in table of contents

% --- fancyhdr package for fancy headers ---
\usepackage{fancyhdr}
\fancyhf{} % sets both header and footer to nothing
\renewcommand{\headrulewidth}{0pt}
\fancyfoot[LE,RO]{\thepage}
% Ensure copyright on titlepage (article style) and chapter pages (book style)
\fancypagestyle{plain}{
  \fancyhf{}
  \fancyfoot[C]{{\footnotesize \copyright\ 1999-2019, Morten Hjorth-Jensen. Released under CC Attribution-NonCommercial 4.0 license}}
%  \renewcommand{\footrulewidth}{0mm}
  \renewcommand{\headrulewidth}{0mm}
}
% Ensure copyright on titlepages with \thispagestyle{empty}
\fancypagestyle{empty}{
  \fancyhf{}
  \fancyfoot[C]{{\footnotesize \copyright\ 1999-2019, Morten Hjorth-Jensen. Released under CC Attribution-NonCommercial 4.0 license}}
  \renewcommand{\footrulewidth}{0mm}
  \renewcommand{\headrulewidth}{0mm}
}

\pagestyle{fancy}


\usepackage[framemethod=TikZ]{mdframed}

% --- begin definitions of admonition environments ---

% --- end of definitions of admonition environments ---

% prevent orhpans and widows
\clubpenalty = 10000
\widowpenalty = 10000

% --- end of standard preamble for documents ---


% insert custom LaTeX commands...

\raggedbottom
\makeindex
\usepackage[totoc]{idxlayout}   % for index in the toc
\usepackage[nottoc]{tocbibind}  % for references/bibliography in the toc

%-------------------- end preamble ----------------------

\begin{document}

% matching end for #ifdef PREAMBLE

\newcommand{\exercisesection}[1]{\subsection*{#1}}


% ------------------- main content ----------------------



% ----------------- title -------------------------

\thispagestyle{empty}

\begin{center}
{\LARGE\bf
\begin{spacing}{1.25}
Data Analysis and Machine Learning: Preprocessing and Dimensionality Reduction
\end{spacing}
}
\end{center}

% ----------------- author(s) -------------------------

\begin{center}
{\bf Morten Hjorth-Jensen${}^{1, 2}$} \\ [0mm]
\end{center}

\begin{center}
% List of all institutions:
\centerline{{\small ${}^1$Department of Physics, University of Oslo}}
\centerline{{\small ${}^2$Department of Physics and Astronomy and National Superconducting Cyclotron Laboratory, Michigan State University}}
\end{center}
    
% ----------------- end author(s) -------------------------

% --- begin date ---
\begin{center}
Oct 22, 2019
\end{center}
% --- end date ---

\vspace{1cm}


% !split
\subsection*{Reducing the number of degrees of freedom, overarching view}

% --- begin paragraph admon ---
\paragraph{}

Many Machine Learning problems involve thousands or even millions of
features for each training instance. Not only does this make training
extremely slow, it can also make it much harder to find a good
solution, as we will see. This problem is often referred to as the
curse of dimensionality.  Fortunately, in real-world problems, it is
often possible to reduce the number of features considerably, turning
an intractable problem into a tractable one.

Here we will discuss some of the most popular dimensionality reduction
techniques: the principal component analysis PCA, Kernel PCA, and
Locally Linear Embedding (LLE).  Furthermore, we will start by looking
at some simple preprocessing of the data which allow us to rescale the
data.
% --- end paragraph admon ---




% !split
\subsection*{Preprocessing our data}

% --- begin paragraph admon ---
\paragraph{}

Before we proceed however, we will discuss how to preprocess our
data. Till now and in connection with our previous examples we have
not met so many cases where we are too sensitive to the scaling of our
data. Normally the data may need a rescaling and/or may be sensitive
to extreme values. Scaling the data renders our inputs much more
suitable for the algorithms we want to employ.

\textbf{Scikit-Learn} has several functions which allow us to rescale the
data, normally resulting in much better results in terms of various
accuracy scores.  The \textbf{StandardScaler} function in \textbf{Scikit-Learn}
ensures that for each feature/predictor we study the mean value is
zero and the variance is one (every column in the design/feature
matrix).  This scaling has the drawback that it does not ensure that
we have a particular maximum or minimum in our data set. Another
function included in \textbf{Scikit-Learn} is the \textbf{MinMaxScaler} which
ensures that all features are exactly between $0$ and $1$. The

% !split
\subsection*{More preprocessing}


The \textbf{Normalizer} scales each data
point such that the feature vector has a euclidean length of one. In other words, it
projects a data point on the circle (or sphere in the case of higher dimensions) with a
radius of 1. This means every data point is scaled by a different number (by the
inverse of it’s length).
This normalization is often used when only the direction (or angle) of the data matters,
not the length of the feature vector.

The \textbf{RobustScaler} works similarly to the StandardScaler in that it
ensures statistical properties for each feature that guarantee that
they are on the same scale. However, the RobustScaler uses the median
and quartiles, instead of mean and variance. This makes the
RobustScaler ignore data points that are very different from the rest
(like measurement errors). These odd data points are also called
outliers, and might often lead to trouble for other scaling
techniques.
% --- end paragraph admon ---



% !split
\subsection*{Simple preprocessing examples, Franke function and regression}

\begin{minted}[fontsize=\fontsize{9pt}{9pt},linenos=false,mathescape,baselinestretch=1.0,fontfamily=tt,xleftmargin=7mm]{python}
# Common imports
import os
import numpy as np
import pandas as pd
import matplotlib.pyplot as plt
import sklearn.linear_model as skl
from sklearn.metrics import mean_squared_error
from sklearn.model_selection import  train_test_split
from sklearn.preprocessing import MinMaxScaler, StandardScaler, Normalizer
from sklearn.svm import SVR

# Where to save the figures and data files
PROJECT_ROOT_DIR = "Results"
FIGURE_ID = "Results/FigureFiles"
DATA_ID = "DataFiles/"

if not os.path.exists(PROJECT_ROOT_DIR):
    os.mkdir(PROJECT_ROOT_DIR)

if not os.path.exists(FIGURE_ID):
    os.makedirs(FIGURE_ID)

if not os.path.exists(DATA_ID):
    os.makedirs(DATA_ID)

def image_path(fig_id):
    return os.path.join(FIGURE_ID, fig_id)

def data_path(dat_id):
    return os.path.join(DATA_ID, dat_id)

def save_fig(fig_id):
    plt.savefig(image_path(fig_id) + ".png", format='png')


def FrankeFunction(x,y):
	term1 = 0.75*np.exp(-(0.25*(9*x-2)**2) - 0.25*((9*y-2)**2))
	term2 = 0.75*np.exp(-((9*x+1)**2)/49.0 - 0.1*(9*y+1))
	term3 = 0.5*np.exp(-(9*x-7)**2/4.0 - 0.25*((9*y-3)**2))
	term4 = -0.2*np.exp(-(9*x-4)**2 - (9*y-7)**2)
	return term1 + term2 + term3 + term4


def create_X(x, y, n ):
	if len(x.shape) > 1:
		x = np.ravel(x)
		y = np.ravel(y)

	N = len(x)
	l = int((n+1)*(n+2)/2)		# Number of elements in beta
	X = np.ones((N,l))

	for i in range(1,n+1):
		q = int((i)*(i+1)/2)
		for k in range(i+1):
			X[:,q+k] = (x**(i-k))*(y**k)

	return X


# Making meshgrid of datapoints and compute Franke's function
n = 5
N = 1000
x = np.sort(np.random.uniform(0, 1, N))
y = np.sort(np.random.uniform(0, 1, N))
z = FrankeFunction(x, y)
X = create_X(x, y, n=n)    
# split in training and test data
X_train, X_test, y_train, y_test = train_test_split(X,z,test_size=0.2)


svm = SVR(gamma='auto',C=10.0)
svm.fit(X_train, y_train)

# The mean squared error and R2 score
print("MSE before scaling: {:.2f}".format(mean_squared_error(svm.predict(X_test), y_test)))
print("R2 score before scaling {:.2f}".format(svm.score(X_test,y_test)))

scaler = StandardScaler()
scaler.fit(X_train)
X_train_scaled = scaler.transform(X_train)
X_test_scaled = scaler.transform(X_test)

print("Feature min values before scaling:\n {}".format(X_train.min(axis=0)))
print("Feature max values before scaling:\n {}".format(X_train.max(axis=0)))

print("Feature min values after scaling:\n {}".format(X_train_scaled.min(axis=0)))
print("Feature max values after scaling:\n {}".format(X_train_scaled.max(axis=0)))

svm = SVR(gamma='auto',C=10.0)
svm.fit(X_train_scaled, y_train)

print("MSE after  scaling: {:.2f}".format(mean_squared_error(svm.predict(X_test_scaled), y_test)))
print("R2 score for  scaled data: {:.2f}".format(svm.score(X_test_scaled,y_test)))

\end{minted}



% !split
\subsection*{Simple preprocessing examples, breast cancer data and classification, Support Vector Machines}

We show here how we can use a simple regression case on the breast
cancer data using support vector machines (SVM) as algorithm for
classification.


\begin{minted}[fontsize=\fontsize{9pt}{9pt},linenos=false,mathescape,baselinestretch=1.0,fontfamily=tt,xleftmargin=7mm]{python}
import matplotlib.pyplot as plt
import numpy as np
from sklearn.model_selection import  train_test_split 
from sklearn.datasets import load_breast_cancer
from sklearn.svm import SVC
cancer = load_breast_cancer()

X_train, X_test, y_train, y_test = train_test_split(cancer.data,cancer.target,random_state=0)
print(X_train.shape)
print(X_test.shape)

svm = SVC(C=100)
svm.fit(X_train, y_train)
print("Test set accuracy: {:.2f}".format(svm.score(X_test,y_test)))

from sklearn.preprocessing import MinMaxScaler, StandardScaler
scaler = MinMaxScaler()
scaler.fit(X_train)
X_train_scaled = scaler.transform(X_train)
X_test_scaled = scaler.transform(X_test)

print("Feature min values before scaling:\n {}".format(X_train.min(axis=0)))
print("Feature max values before scaling:\n {}".format(X_train.max(axis=0)))

print("Feature min values before scaling:\n {}".format(X_train_scaled.min(axis=0)))
print("Feature max values before scaling:\n {}".format(X_train_scaled.max(axis=0)))


svm.fit(X_train_scaled, y_train)
print("Test set accuracy scaled data with Min-Max scaling: {:.2f}".format(svm.score(X_test_scaled,y_test)))

scaler = StandardScaler()
scaler.fit(X_train)
X_train_scaled = scaler.transform(X_train)
X_test_scaled = scaler.transform(X_test)

svm.fit(X_train_scaled, y_train)
print("Test set accuracy scaled data with Standar Scaler: {:.2f}".format(svm.score(X_test_scaled,y_test)))

\end{minted}

% !split
\subsection*{More on Cancer Data, now with Logistic Regression}


\begin{minted}[fontsize=\fontsize{9pt}{9pt},linenos=false,mathescape,baselinestretch=1.0,fontfamily=tt,xleftmargin=7mm]{python}
import matplotlib.pyplot as plt
import numpy as np
from sklearn.model_selection import  train_test_split 
from sklearn.datasets import load_breast_cancer
from sklearn.linear_model import LogisticRegression
cancer = load_breast_cancer()

# Set up training data
X_train, X_test, y_train, y_test = train_test_split(cancer.data,cancer.target,random_state=0)
logreg = LogisticRegression()
logreg.fit(X_train, y_train)
print("Test set accuracy: {:.2f}".format(logreg.score(X_test,y_test)))

# Scale data
from sklearn.preprocessing import StandardScaler
scaler = StandardScaler()
scaler.fit(X_train)
X_train_scaled = scaler.transform(X_train)
X_test_scaled = scaler.transform(X_test)
logreg.fit(X_train_scaled, y_train)
print("Test set accuracy scaled data: {:.2f}".format(logreg.score(X_test_scaled,y_test)))

\end{minted}




% !split
\subsection*{Why should we think of reducing the dimensionality}

In addition to the plot of the features, we study now also the covariance (or rather the correlation matrix).
We use also \textbf{Pandas} to compute the correlation matrix.
\begin{minted}[fontsize=\fontsize{9pt}{9pt},linenos=false,mathescape,baselinestretch=1.0,fontfamily=tt,xleftmargin=7mm]{python}
import matplotlib.pyplot as plt
import numpy as np
from sklearn.model_selection import  train_test_split 
from sklearn.datasets import load_breast_cancer
from sklearn.linear_model import LogisticRegression
cancer = load_breast_cancer()
import pandas as pd
# Making a data frame
cancerpd = pd.DataFrame(cancer.data, columns=cancer.feature_names)

fig, axes = plt.subplots(15,2,figsize=(10,20))
malignant = cancer.data[cancer.target == 0]
benign = cancer.data[cancer.target == 1]
ax = axes.ravel()

for i in range(30):
    _, bins = np.histogram(cancer.data[:,i], bins =50)
    ax[i].hist(malignant[:,i], bins = bins, alpha = 0.5)
    ax[i].hist(benign[:,i], bins = bins, alpha = 0.5)
    ax[i].set_title(cancer.feature_names[i])
    ax[i].set_yticks(())
ax[0].set_xlabel("Feature magnitude")
ax[0].set_ylabel("Frequency")
ax[0].legend(["Malignant", "Benign"], loc ="best")
fig.tight_layout()
plt.show()

import seaborn as sns
correlation_matrix = cancerpd.corr().round(1)
# use the heatmap function from seaborn to plot the correlation matrix
# annot = True to print the values inside the square
sns.heatmap(data=correlation_matrix, annot=True)
plt.show()

#print eigvalues of correlation matrix
EigValues, EigVectors = np.linalg.eig(correlation_matrix)
print(EigValues)
\end{minted}

In the above example we note two things. In the first plot we display
the overlap of benign and malignant tumors as functions of the various
features in the Wisconsing breast cancer data set. We see that for
some of the features we can distinguish clearly the benign and
malignant cases while for other features we cannot. This can point to
us which features may be of greater interest when we wish to classify
a benign or not benign tumour.

In the second figure we have computed the so-called correlation
matrix, which in our case with thirty features becomes a $30\times 30$
matrix.

We constructed this matrix using \textbf{pandas} via the statements
\begin{minted}[fontsize=\fontsize{9pt}{9pt},linenos=false,mathescape,baselinestretch=1.0,fontfamily=tt,xleftmargin=7mm]{python}
cancerpd = pd.DataFrame(cancer.data, columns=cancer.feature_names)
\end{minted}
and then
\begin{minted}[fontsize=\fontsize{9pt}{9pt},linenos=false,mathescape,baselinestretch=1.0,fontfamily=tt,xleftmargin=7mm]{python}
correlation_matrix = cancerpd.corr().round(1)
\end{minted}

Diagonalizing this matrix we can in turn say something about which
features are of relevance and which are not. But before we proceed we
need to define covariance and correlation matrices. This leads us to
the classical Principal Component Analysis (PCA) theorem with
applications.



% !split
\subsection*{Basic ideas of the Principal Component Analysis (PCA)}

We have a data set defined by a design/feature matrix $\bm{X}$ (see below for its definition) 
\begin{itemize}
\item Each data point is determined by $p$ extrinsic (measurement) variables

\item We may want to ask the following question: Are there fewer intrinsic variables (say $d << p$) that still approximately describe the data?

\item If so, these intrinsic variables may tell us something important and finding these intrinsic variables is what dimension reduction methods do. 
\end{itemize}

\noindent
% !split
\subsection*{Introducing the Covariance and Correlation functions}

Before we discuss the PCA theorem, we need to remind ourselves about
the definition of the covariance and the correlation function.

Suppose we have defined two vectors
$\hat{x}$ and $\hat{y}$ with $n$ elements each. The covariance matrix $\bm{C}$ is defined as 
\[
\bm{C}[\bm{x},\bm{y}] = \begin{bmatrix} \mathrm{cov}[\bm{x},\bm{x}] & \mathrm{cov}[\bm{x},\bm{y}] \\
                              \mathrm{cov}[\bm{y},\bm{x}] & \mathrm{cov}[\bm{y},\bm{y}] \\
             \end{bmatrix},
\]
where for example
\[
\mathrm{cov}[\bm{x},\bm{y}] =\frac{1}{n} \sum_{i=0}^{n-1}(x_i- \overline{x})(y_i- \overline{y}).
\]
With this definition and recalling that the variance is defined as
\[
\mathrm{var}[\bm{x}]=\frac{1}{n} \sum_{i=0}^{n-1}(x_i- \overline{x})^2,
\]
we can rewrite the covariance matrix as 
\[
\bm{C}[\bm{x},\bm{y}] = \begin{bmatrix} \mathrm{var}[\bm{x}] & \mathrm{cov}[\bm{x},\bm{y}] \\
                              \mathrm{cov}[\bm{x},\bm{y}] & \mathrm{var}[\bm{y}] \\
             \end{bmatrix}.
\]

The covariance takes values between zero and infinity and may thus
lead to problems with loss of numerical precision for particularly
large values. It is common to scale the covariance matrix by
introducing instead the correlation matrix defined via the so-called
correlation function

\[
\mathrm{corr}[\bm{x},\bm{y}]=\frac{\mathrm{cov}[\bm{x},\bm{y}]}{\sqrt{\mathrm{var}[\bm{x}] \mathrm{var}[\bm{y}]}}.
\]

The correlation function is then given by values $\mathrm{corr}[\bm{x},\bm{y}]
\in [-1,1]$. This avoids eventual problems with too large values. We
can then define the correlation matrix for the two vectors $\bm{x}$
and $\bm{y}$ as

\[
\bm{K}[\bm{x},\bm{y}] = \begin{bmatrix} 1 & \mathrm{corr}[\bm{x},\bm{y}] \\
                              \mathrm{corr}[\bm{y},\bm{x}] & 1 \\
             \end{bmatrix},
\]

In the above example this is the function we constructed using \textbf{pandas}.

% !split
\subsection*{Correlation Function and Design/Feature Matrix}

In our derivation of the various regression algorithms like Ordinary Least Squares or Ridge regression we defined the design/feature matrix $\bm{X}$ as
\[
\bm{X}=\begin{bmatrix}
x_{0,0} & x_{0,1} & x_{0,2}& \dots & \dots x_{0,p-1}\\
x_{1,0} & x_{1,1} & x_{1,2}& \dots & \dots x_{1,p-1}\\
x_{2,0} & x_{2,1} & x_{2,2}& \dots & \dots x_{2,p-1}\\
\dots & \dots & \dots & \dots \dots & \dots \\
x_{n-2,0} & x_{n-2,1} & x_{n-2,2}& \dots & \dots x_{n-2,p-1}\\
x_{n-1,0} & x_{n-1,1} & x_{n-1,2}& \dots & \dots x_{n-1,p-1}\\
\end{bmatrix},
\]
with $\bm{X}\in {\mathbb{R}}^{n\times p}$, with the predictors/features $p$  refering to the column numbers and the
entries $n$ being the row elements.
We can rewrite the design/feature matrix in terms of its column vectors as
\[
\bm{X}=\begin{bmatrix} \bm{x}_0 & \bm{x}_0 & \bm{x}_0 & \dots & \dots & \bm{x}_{p-1}\end{bmatrix},
\]
with a given vector
\[
\bm{x}_i^T = \begin{bmatrix}x_{0,i} & x_{1,i} & x_{2,i}& \dots & \dots x_{n-1,i}\end{bmatrix}.
\]

With these definitions, we can now rewrite our $2\times 2$ correaltion/covariance matrix in terms of a moe general design/feature matrix $\bm{X}\in {\mathbb{R}}^{n\times p}$. This leads to a $p\times p$ covariance matrix for the vectors $\bm{x}_i$ with $i =0,1,\dots,p-1$
\[
\bm{C}[\bm{x}] = \begin{bmatrix}
\mathrm{var}[\bm{x}_0] & \mathrm{cov}[\bm{x}_0,\bm{x}_1]  & \mathrm{cov}[\bm{x}_0,\bm{x}_2] & \dots & \dots & \mathrm{cov}[\bm{x}_0,\bm{x}_{p-1}]\\
\mathrm{cov}[\bm{x}_1,\bm{x}_0] & \mathrm{var}[\bm{x}_1]  & \mathrm{cov}[\bm{x}_1,\bm{x}_2] & \dots & \dots & \mathrm{cov}[\bm{x}_1,\bm{x}_{p-1}]\\
\mathrm{cov}[\bm{x}_2,\bm{x}_0]   & \mathrm{cov}[\bm{x}_2,\bm{x}_1] & \mathrm{var}[\bm{x}_2] & \dots & \dots & \mathrm{cov}[\bm{x}_2,\bm{x}_{p-1}]\\
\dots & \dots & \dots & \dots & \dots & \dots \\
\dots & \dots & \dots & \dots & \dots & \dots \\
\mathrm{cov}[\bm{x}_{p-1},\bm{x}_0]   & \mathrm{cov}[\bm{x}_{p-1},\bm{x}_1] & \mathrm{cov}[\bm{x}_{p-1},\bm{x}_{2}]  & \dots & \dots  & \mathrm{var}[\bm{x}_{p-1}]\\
\end{bmatrix},
\]
and the correlation matrix
\[
\bm{K}[\bm{x}] = \begin{bmatrix}
1 & \mathrm{corr}[\bm{x}_0,\bm{x}_1]  & \mathrm{corr}[\bm{x}_0,\bm{x}_2] & \dots & \dots & \mathrm{corr}[\bm{x}_0,\bm{x}_{p-1}]\\
\mathrm{corr}[\bm{x}_1,\bm{x}_0] & 1  & \mathrm{corr}[\bm{x}_1,\bm{x}_2] & \dots & \dots & \mathrm{corr}[\bm{x}_1,\bm{x}_{p-1}]\\
\mathrm{corr}[\bm{x}_2,\bm{x}_0]   & \mathrm{corr}[\bm{x}_2,\bm{x}_1] & 1 & \dots & \dots & \mathrm{corr}[\bm{x}_2,\bm{x}_{p-1}]\\
\dots & \dots & \dots & \dots & \dots & \dots \\
\dots & \dots & \dots & \dots & \dots & \dots \\
\mathrm{corr}[\bm{x}_{p-1},\bm{x}_0]   & \mathrm{corr}[\bm{x}_{p-1},\bm{x}_1] & \mathrm{corr}[\bm{x}_{p-1},\bm{x}_{2}]  & \dots & \dots  & 1\\
\end{bmatrix},
\]


% !split
\subsection*{Covariance Matrix Examples}


The Numpy function \textbf{np.cov} calculates the covariance elements using
the factor $1/(n-1)$ instead of $1/n$ since it assumes we do not have
the exact mean values.  The following simple function uses the
\textbf{np.vstack} function which takes each vector of dimension $1\times n$
and produces a $2\times n$ matrix $\bm{W}$


\[
\bm{W} = \begin{bmatrix} x_0 & y_0 \\
                          x_1 & y_1 \\
                          x_2 & y_2\\
                          \dots & \dots \\
                          x_{n-2} & y_{n-2}\\
                          x_{n-1} & y_{n-1} & 
             \end{bmatrix},
\]

which in turn is converted into into the $2\times 2$ covariance matrix
$\bm{C}$ via the Numpy function \textbf{np.cov()}. We note that we can also calculate
the mean value of each set of samples $\bm{x}$ etc using the Numpy
function \textbf{np.mean(x)}. We can also extract the eigenvalues of the
covariance matrix through the \textbf{np.linalg.eig()} function.

\begin{minted}[fontsize=\fontsize{9pt}{9pt},linenos=false,mathescape,baselinestretch=1.0,fontfamily=tt,xleftmargin=7mm]{python}
# Importing various packages
import numpy as np
n = 100
x = np.random.normal(size=n)
print(np.mean(x))
y = 4+3*x+np.random.normal(size=n)
print(np.mean(y))
W = np.vstack((x, y))
C = np.cov(W)
print(C)
\end{minted}

% !split
\subsection*{Correlation Matrix}

The previous example can be converted into the correlation matrix by
simply scaling the matrix elements with the variances.  We should also
subtract the mean values for each column. This leads to the following
code which sets up the correlations matrix for the previous example in
a more brute force way. Here we scale the mean values for each column of the design matrix, calculate the relevant mean values and variances and then finally set up the $2\times 2$ correlation matrix (since we have only two vectors). 

\begin{minted}[fontsize=\fontsize{9pt}{9pt},linenos=false,mathescape,baselinestretch=1.0,fontfamily=tt,xleftmargin=7mm]{python}
import numpy as np
n = 100
# define two vectors                                                                                           
x = np.random.random(size=n)
y = 4+3*x+np.random.normal(size=n)
#scaling the x and y vectors                                                                                   
x = x - np.mean(x)
y = y - np.mean(y)
variance_x = np.sum(x@x)/n
variance_y = np.sum(y@y)/n
print(variance_x)
print(variance_y)
cov_xy = np.sum(x@y)/n
cov_xx = np.sum(x@x)/n
cov_yy = np.sum(y@y)/n
C = np.zeros((2,2))
C[0,0]= cov_xx/variance_x
C[1,1]= cov_yy/variance_y
C[0,1]= cov_xy/np.sqrt(variance_y*variance_x)
C[1,0]= C[0,1]
print(C)
\end{minted}

We see that the matrix elements along the diagonal are one as they
should be and that the matrix is symmetric. Furthermore, diagonalizing
this matrix we easily see that it is a positive definite matrix.

The above procedure with \textbf{numpy} can be made more compact if we use \textbf{pandas}.

% !split
\subsection*{Correlation Matrix with Pandas}

We whow here how we can set up the correlation matrix using \textbf{pandas}, as done in this simple code
\begin{minted}[fontsize=\fontsize{9pt}{9pt},linenos=false,mathescape,baselinestretch=1.0,fontfamily=tt,xleftmargin=7mm]{python}
import numpy as np
import pandas as pd
n = 10
x = np.random.normal(size=n)
x = x - np.mean(x)
y = 4+3*x+np.random.normal(size=n)
y = y - np.mean(y)
X = (np.vstack((x, y))).T
print(X)
Xpd = pd.DataFrame(X)
print(Xpd)
correlation_matrix = Xpd.corr()
print(correlation_matrix)
\end{minted}


We expand this model to the Franke function discussed above.

% !split
\subsection*{Correlation Matrix with Pandas and the Franke function}

\begin{minted}[fontsize=\fontsize{9pt}{9pt},linenos=false,mathescape,baselinestretch=1.0,fontfamily=tt,xleftmargin=7mm]{python}
# Common imports
import numpy as np
import pandas as pd


def FrankeFunction(x,y):
	term1 = 0.75*np.exp(-(0.25*(9*x-2)**2) - 0.25*((9*y-2)**2))
	term2 = 0.75*np.exp(-((9*x+1)**2)/49.0 - 0.1*(9*y+1))
	term3 = 0.5*np.exp(-(9*x-7)**2/4.0 - 0.25*((9*y-3)**2))
	term4 = -0.2*np.exp(-(9*x-4)**2 - (9*y-7)**2)
	return term1 + term2 + term3 + term4


def create_X(x, y, n ):
	if len(x.shape) > 1:
		x = np.ravel(x)
		y = np.ravel(y)

	N = len(x)
	l = int((n+1)*(n+2)/2)		# Number of elements in beta
	X = np.ones((N,l))

	for i in range(1,n+1):
		q = int((i)*(i+1)/2)
		for k in range(i+1):
			X[:,q+k] = (x**(i-k))*(y**k)

	return X


# Making meshgrid of datapoints and compute Franke's function
n = 4
N = 100
x = np.sort(np.random.uniform(0, 1, N))
y = np.sort(np.random.uniform(0, 1, N))
z = FrankeFunction(x, y)
X = create_X(x, y, n=n)    

Xpd = pd.DataFrame(X)
# subtract the mean values and set up the covariance matrix
Xpd = Xpd - Xpd.mean()
covariance_matrix = Xpd.cov()
print(covariance_matrix)
\end{minted}

We note here that the covariance is zero for the first rows and
columns since all matrix elements in the design matrix were set to one
(we are fitting the function in terms of a polynomial of degree $n$).

This means that the variance for these elements will be zero and will
cause problems when we set up the correlation matrix.  We can simply
drop these elements as follows and then construct the correlation
matrix. 


% !split
\subsection*{Rewriting the Covariance and/or Correlation Matrix}

We can rewrite the covariance matrix in a more compact form in terms of the design/feature matrix $\bm{X}$ as 
\[
\bm{C}[\bm{x}] = \frac{1}{n}\bm{X}\bm{X}^T= \mathbb{E}[\bm{X}\bm{X}^T].
\]

To see this let us simply look at a design matrix $\bm{X}\in {\mathbb{R}}^{2\times 2}$
\[
\bm{X}=\begin{bmatrix}
x_{00} & x_{01}\\
x_{10} & x_{11}\\
\end{bmatrix}=\begin{bmatrix}
\bm{x}_{0} & \bm{x}_{1}\\
\end{bmatrix}.
\]

If we then compute the expectation value
\[
\mathbb{E}[\bm{X}\bm{X}^T] = \frac{1}{n}\bm{X}\bm{X}^T=\begin{bmatrix}
x_{00}^2+x_{01}^2 & x_{00}x_{10}+x_{01}x_{11}\\
x_{10}x_{00}+x_{01}x_{11} & x_{10}^2+x_{11}^2\\
\end{bmatrix},
\]
which is just 
\[
\bm{C}[\bm{x}_0,\bm{x}_1] = \bm{C}[\bm{x}]=\begin{bmatrix} \mathrm{var}[\bm{x}_0] & \mathrm{cov}[\bm{x}_0,\bm{x}_1] \\
                              \mathrm{cov}[\bm{x}_1,\bm{x}_0] & \mathrm{var}[\bm{x}_1] \\
             \end{bmatrix},
\]
where we wrote $$\bm{C}[\bm{x}_0,\bm{x}_1] = \bm{C}[\bm{x}]$$ to indicate that this the covariance of the vectors $\bm{x}$ of the design/feature matrix $\bm{X}$.

It is easy to generalize this to a matrix $\bm{X}\in {\mathbb{R}}^{n\times p}$.


% !split
\subsection*{Towards the PCA theorem}

We have that the covariance matrix (the correlation matrix involves a simple rescaling) is given as
\[
\bm{C}[\bm{x}] = \frac{1}{n}\bm{X}\bm{X}^T= \mathbb{E}[\bm{X}\bm{X}^T].
\]
Let us now assume that we can perform a series of orthogonal transformations where we employ some orthogonal matrices $\bm{S}$.
These matrices are defined as $\bm{S}\in {\mathbb{R}}^{p\times p}$ and obey the orthogonality requirements $\bm{S}\bm{S}^T=\bm{S}^T\bm{S}=\bm{I}$. The matrix can be written out in terms of the column vectors $\bm{s}_i$ as $\bm{S}=[\bm{s}_0,\bm{s}_1,\dots,\bm{s}_{p-1}]$ and $\bm{s}_i \in {\mathbb{R}}^{p}$.

Assume also that there is a transformation $\bm{S}\bm{C}[\bm{x}]\bm{S}^T=\bm{C}[\bm{y}]$ such that the new matrix $\bm{C}[\bm{y}]$ is diagonal with elements $[\lambda_0,\lambda_1,\lambda_2,\dots,\lambda_{p-1}]$.  

That is we have
\[
\bm{C}[\bm{y}] = \mathbb{E}[\bm{S}\bm{X}\bm{X}^T\bm{S}^T]=\bm{S}\bm{C}[\bm{x}]\bm{S}^T,
\]
since the matrix $\bm{S}$ is not a data dependent matrix.   Multiplying with $\bm{S}^T$ from the left we have
\[
\bm{S}^T\bm{C}[\bm{y}] = \bm{C}[\bm{x}]\bm{S}^T,
\]
and since $\bm{C}[\bm{y}]$ is diagonal we have for a given eigenvalue $i$ of the covariance matrix that

\[
\bm{S}^T_i\lambda_i = \bm{C}[\bm{x}]\bm{S}^T_i.
\]

In the derivation of the PCA theorem we will assume that the eigenvalues are ordered in descending order, that is
$\lambda_0 > \lambda_1 > \dots > \lambda_{p-1}$. 

% !split
\subsection*{Classical PCA Theorem}



% !split
\subsection*{Prof of the PCA Theorem}






% !split
\subsection*{Getting started with PCA}


\begin{minted}[fontsize=\fontsize{9pt}{9pt},linenos=false,mathescape,baselinestretch=1.0,fontfamily=tt,xleftmargin=7mm]{python}
# Now add PCA
from sklearn.decomposition import PCA
pca = PCA(n_components = 2)
pca.fit(X_train_scaled)

X_pca = pca.transform(X_train_scaled)
\end{minted}



% !split
\subsection*{Principal Component Analysis}

% --- begin paragraph admon ---
\paragraph{}
Principal Component Analysis (PCA) is by far the most popular dimensionality reduction algorithm.
First it identifies the hyperplane that lies closest to the data, and then it projects the data onto it.

The following Python code uses NumPy’s \textbf{svd()} function to obtain all the principal components of the
training set, then extracts the first two principal components
\begin{minted}[fontsize=\fontsize{9pt}{9pt},linenos=false,mathescape,baselinestretch=1.0,fontfamily=tt,xleftmargin=7mm]{python}
X_centered = X - X.mean(axis=0)
U, s, V = np.linalg.svd(X_centered)
c1 = V.T[:, 0]
c2 = V.T[:, 1]
\end{minted}

PCA assumes that the dataset is centered around the origin. Scikit-Learn’s PCA classes take care of centering
the data for you. However, if you implement PCA yourself (as in the preceding example), or if you use other libraries, don’t
forget to center the data first.

Once you have identified all the principal components, you can reduce the dimensionality of the dataset
down to $d$ dimensions by projecting it onto the hyperplane defined by the first $d$ principal components.
Selecting this hyperplane ensures that the projection will preserve as much variance as possible. 
\begin{minted}[fontsize=\fontsize{9pt}{9pt},linenos=false,mathescape,baselinestretch=1.0,fontfamily=tt,xleftmargin=7mm]{python}
W2 = V.T[:, :2]
X2D = X_centered.dot(W2)
\end{minted}

% !split 
\subsection*{PCA and scikit-learn}

Scikit-Learn’s PCA class implements PCA using SVD decomposition just like we did before. The
following code applies PCA to reduce the dimensionality of the dataset down to two dimensions (note
that it automatically takes care of centering the data):
\begin{minted}[fontsize=\fontsize{9pt}{9pt},linenos=false,mathescape,baselinestretch=1.0,fontfamily=tt,xleftmargin=7mm]{python}
from sklearn.decomposition import PCA
pca = PCA(n_components = 2)
X2D = pca.fit_transform(X)
\end{minted}
After fitting the PCA transformer to the dataset, you can access the principal components using the
components variable (note that it contains the PCs as horizontal vectors, so, for example, the first
principal component is equal to 
\begin{minted}[fontsize=\fontsize{9pt}{9pt},linenos=false,mathescape,baselinestretch=1.0,fontfamily=tt,xleftmargin=7mm]{python}
pca.components_.T[:, 0]).
\end{minted}
Another very useful piece of information is the explained variance ratio of each principal component,
available via the $explained\_variance\_ratio$ variable. It indicates the proportion of the dataset’s
variance that lies along the axis of each principal component. 
More material to come here.

% !split
\subsection*{More on the PCA}

Instead of arbitrarily choosing the number of dimensions to reduce down to, it is generally preferable to
choose the number of dimensions that add up to a sufficiently large portion of the variance (e.g., 95\%).
Unless, of course, you are reducing dimensionality for data visualization — in that case you will
generally want to reduce the dimensionality down to 2 or 3.
The following code computes PCA without reducing dimensionality, then computes the minimum number
of dimensions required to preserve 95\% of the training set’s variance:
\begin{minted}[fontsize=\fontsize{9pt}{9pt},linenos=false,mathescape,baselinestretch=1.0,fontfamily=tt,xleftmargin=7mm]{python}
pca = PCA()
pca.fit(X)
cumsum = np.cumsum(pca.explained_variance_ratio_)
d = np.argmax(cumsum >= 0.95) + 1
\end{minted}
You could then set $n\_components=d$ and run PCA again. However, there is a much better option: instead
of specifying the number of principal components you want to preserve, you can set $n\_components$ to be
a float between 0.0 and 1.0, indicating the ratio of variance you wish to preserve:
\begin{minted}[fontsize=\fontsize{9pt}{9pt},linenos=false,mathescape,baselinestretch=1.0,fontfamily=tt,xleftmargin=7mm]{python}
pca = PCA(n_components=0.95)
X_reduced = pca.fit_transform(X)
\end{minted}

% !split
\subsection*{Incremental PCA}

One problem with the preceding implementation of PCA is that it requires the whole training set to fit in
memory in order for the SVD algorithm to run. Fortunately, Incremental PCA (IPCA) algorithms have
been developed: you can split the training set into mini-batches and feed an IPCA algorithm one minibatch
at a time. This is useful for large training sets, and also to apply PCA online (i.e., on the fly, as new
instances arrive).

% !split
\subsection*{Randomized PCA}

Scikit-Learn offers yet another option to perform PCA, called Randomized PCA. This is a stochastic
algorithm that quickly finds an approximation of the first d principal components. Its computational
complexity is $O(m \times d^2)+O(d^3)$, instead of $O(m \times n^2) + O(n^3)$, so it is dramatically faster than the
previous algorithms when $d$ is much smaller than $n$.
% --- end paragraph admon ---




% !split
\subsection*{Kernel PCA}

% --- begin paragraph admon ---
\paragraph{}

The kernel trick is a mathematical technique that implicitly maps instances into a
very high-dimensional space (called the feature space), enabling nonlinear classification and regression
with Support Vector Machines. Recall that a linear decision boundary in the high-dimensional feature
space corresponds to a complex nonlinear decision boundary in the original space.
It turns out that the same trick can be applied to PCA, making it possible to perform complex nonlinear
projections for dimensionality reduction. This is called Kernel PCA (kPCA). It is often good at
preserving clusters of instances after projection, or sometimes even unrolling datasets that lie close to a
twisted manifold.
For example, the following code uses Scikit-Learn’s KernelPCA class to perform kPCA with an
\begin{minted}[fontsize=\fontsize{9pt}{9pt},linenos=false,mathescape,baselinestretch=1.0,fontfamily=tt,xleftmargin=7mm]{python}
from sklearn.decomposition import KernelPCA
rbf_pca = KernelPCA(n_components = 2, kernel="rbf", gamma=0.04)
X_reduced = rbf_pca.fit_transform(X)
\end{minted}
% --- end paragraph admon ---

 


% !split
\subsection*{LLE}

Locally Linear Embedding (LLE) is another very powerful nonlinear dimensionality reduction
(NLDR) technique. It is a Manifold Learning technique that does not rely on projections like the previous
algorithms. In a nutshell, LLE works by first measuring how each training instance linearly relates to its
closest neighbors (c.n.), and then looking for a low-dimensional representation of the training set where
these local relationships are best preserved (more details shortly). 



% !split
\subsection*{Other techniques}


There are many other dimensionality reduction techniques, several of which are available in Scikit-Learn.

Here are some of the most popular:
\begin{itemize}
\item \textbf{Multidimensional Scaling (MDS)} reduces dimensionality while trying to preserve the distances between the instances.

\item \textbf{Isomap} creates a graph by connecting each instance to its nearest neighbors, then reduces dimensionality while trying to preserve the geodesic distances between the instances.

\item \textbf{t-Distributed Stochastic Neighbor Embedding} (t-SNE) reduces dimensionality while trying to keep similar instances close and dissimilar instances apart. It is mostly used for visualization, in particular to visualize clusters of instances in high-dimensional space (e.g., to visualize the MNIST images in 2D).

\item Linear Discriminant Analysis (LDA) is actually a classification algorithm, but during training it learns the most discriminative axes between the classes, and these axes can then be used to define a hyperplane onto which to project the data. The benefit is that the projection will keep classes as far apart as possible, so LDA is a good technique to reduce dimensionality before running another classification algorithm such as a Support Vector Machine (SVM) classifier discussed in the SVM lectures.
\end{itemize}

\noindent
Here are other examples where we use the \textbf{DataFrame} functionality to handle arrays, now with more interesting features for us, namely numbers. We set up a matrix 
of dimensionality $10\times 5$ and compute the mean value and standard deviation of each column. Similarly, we can perform mathematial operations like squaring the matrix elements and many other operations. 
\begin{minted}[fontsize=\fontsize{9pt}{9pt},linenos=false,mathescape,baselinestretch=1.0,fontfamily=tt,xleftmargin=7mm]{python}
import numpy as np
import pandas as pd
from IPython.display import display
np.random.seed(100)
# setting up a 10 x 5 matrix
rows = 10
cols = 5
a = np.random.randn(rows,cols)
df = pd.DataFrame(a)
display(df)
print(df.mean())
print(df.std())
display(df**2)
\end{minted}

Thereafter we can select specific columns only and plot final results
\begin{minted}[fontsize=\fontsize{9pt}{9pt},linenos=false,mathescape,baselinestretch=1.0,fontfamily=tt,xleftmargin=7mm]{python}
df.columns = ['First', 'Second', 'Third', 'Fourth', 'Fifth']
df.index = np.arange(10)

display(df)
print(df['Second'].mean() )
print(df.info())
print(df.describe())

from pylab import plt, mpl
plt.style.use('seaborn')
mpl.rcParams['font.family'] = 'serif'

df.cumsum().plot(lw=2.0, figsize=(10,6))
plt.show()


df.plot.bar(figsize=(10,6), rot=15)
plt.show()
\end{minted}
We can produce a $4\times 4$ matrix
\begin{minted}[fontsize=\fontsize{9pt}{9pt},linenos=false,mathescape,baselinestretch=1.0,fontfamily=tt,xleftmargin=7mm]{python}
b = np.arange(16).reshape((4,4))
print(b)
df1 = pd.DataFrame(b)
print(df1)
\end{minted}
and many other operations. 


% ------------------- end of main content ---------------

\end{document}

