
% LaTeX Beamer file automatically generated from DocOnce
% https://github.com/hplgit/doconce

%-------------------- begin beamer-specific preamble ----------------------

\documentclass{beamer}

\usetheme{red_plain}
\usecolortheme{default}

% turn off the almost invisible, yet disturbing, navigation symbols:
\setbeamertemplate{navigation symbols}{}

% Examples on customization:
%\usecolortheme[named=RawSienna]{structure}
%\usetheme[height=7mm]{Rochester}
%\setbeamerfont{frametitle}{family=\rmfamily,shape=\itshape}
%\setbeamertemplate{items}[ball]
%\setbeamertemplate{blocks}[rounded][shadow=true]
%\useoutertheme{infolines}
%
%\usefonttheme{}
%\useinntertheme{}
%
%\setbeameroption{show notes}
%\setbeameroption{show notes on second screen=right}

% fine for B/W printing:
%\usecolortheme{seahorse}

\usepackage{pgf}
\usepackage{graphicx}
\usepackage{epsfig}
\usepackage{relsize}

\usepackage{fancybox}  % make sure fancybox is loaded before fancyvrb

\usepackage{fancyvrb}
%\usepackage{minted} % requires pygments and latex -shell-escape filename
%\usepackage{anslistings}
%\usepackage{listingsutf8}

\usepackage{amsmath,amssymb,bm}
%\usepackage[latin1]{inputenc}
\usepackage[T1]{fontenc}
\usepackage[utf8]{inputenc}
\usepackage{colortbl}
\usepackage[english]{babel}
\usepackage{tikz}
\usepackage{framed}
% Use some nice templates
\beamertemplatetransparentcovereddynamic

% --- begin table of contents based on sections ---
% Delete this, if you do not want the table of contents to pop up at
% the beginning of each section:
% (Only section headings can enter the table of contents in Beamer
% slides generated from DocOnce source, while subsections are used
% for the title in ordinary slides.)
\AtBeginSection[]
{
  \begin{frame}<beamer>[plain]
  \frametitle{}
  %\frametitle{Outline}
  \tableofcontents[currentsection]
  \end{frame}
}
% --- end table of contents based on sections ---

% If you wish to uncover everything in a step-wise fashion, uncomment
% the following command:

%\beamerdefaultoverlayspecification{<+->}

\newcommand{\shortinlinecomment}[3]{\note{\textbf{#1}: #2}}
\newcommand{\longinlinecomment}[3]{\shortinlinecomment{#1}{#2}{#3}}

\definecolor{linkcolor}{rgb}{0,0,0.4}
\hypersetup{
    colorlinks=true,
    linkcolor=linkcolor,
    urlcolor=linkcolor,
    pdfmenubar=true,
    pdftoolbar=true,
    bookmarksdepth=3
    }
\setlength{\parskip}{0pt}  % {1em}

\newenvironment{doconceexercise}{}{}
\newcounter{doconceexercisecounter}
\newenvironment{doconce:movie}{}{}
\newcounter{doconce:movie:counter}

\newcommand{\subex}[1]{\noindent\textbf{#1}}  % for subexercises: a), b), etc

\logo{{\tiny \copyright\ 1999-2018, Morten Hjorth-Jensen. Released under CC Attribution-NonCommercial 4.0 license}}

%-------------------- end beamer-specific preamble ----------------------

% Add user's preamble




% insert custom LaTeX commands...

\raggedbottom
\makeindex

%-------------------- end preamble ----------------------

\begin{document}

% matching end for #ifdef PREAMBLE

\newcommand{\exercisesection}[1]{\subsection*{#1}}



% ------------------- main content ----------------------



% ----------------- title -------------------------

\title{Data Analysis and Machine Learning: Machine learning with Gaussian Processes}

% ----------------- author(s) -------------------------

\author{Christian Forssén\inst{1}
\and
Morten Hjorth-Jensen\inst{2,3}}
\institute{Department of Physics, Chalmers University of Technology, Sweden\inst{1}
\and
Department of Physics, University of Oslo\inst{2}
\and
Department of Physics and Astronomy and National Superconducting Cyclotron Laboratory, Michigan State University\inst{3}}
% ----------------- end author(s) -------------------------

\date{Mar 19, 2018
% <optional titlepage figure>
\ \\ 
{\tiny \copyright\ 1999-2018, Morten Hjorth-Jensen. Released under CC Attribution-NonCommercial 4.0 license}
}

\begin{frame}[plain,fragile]
\titlepage
\end{frame}

\begin{frame}[plain,fragile]
\frametitle{What is a Gaussian Process?}

\begin{itemize}
\item We have considered splines and kernel regression methods. These
\end{itemize}

\noindent
require choice of somewhat arbitrary set of knots.

\begin{itemize}
\item Antoher possibility is to setup a prior distribution for the
  regression function using a \emph{Gaussian Process}.

\item This is a very flexible class of models that has distinct computational
  and theoretical advantages. It can be viewed as a potentially
  infinite-dimensional generalization of Gaussian distributions.

\item See the excellent (and free) book \href{{http://www.gaussianprocess.org/gpml/}}{Gaussian Processes for Machine
  Learning} by Carl Edward
  Rasmussen and Christopher K. I. Williams. 
\end{itemize}

\noindent
\end{frame}

\begin{frame}[plain,fragile]
\frametitle{Gaussian process regression}

\begin{itemize}
\item Realizations from a Gaussian process correspond to random functions

\item Let us first consider an unknown regression function $\mu(x)$ that
  depends on a single, continuous variable $x$.

\item The Gaussian process is written as $\mu \sim \mathrm{GP}(m,k)$, and
  is parametrized in terms of a mean function $m(x)$ and a covariance
  function $k(x,x')$.

\item The GP prior on $\mu$ describes it as a random function for which
  the values at any set of $N$ prespecified points $\{x_i\}_{i=1}^N$
  are a draw from a $N$-dimensional normal distribution
\end{itemize}

\noindent
$$
  \mu(x_1), \ldots \mu(x_N) \sim \mathrm{N}\left( \left( m(x_1),
  \ldots, m(x_N) \right), K(x_1, \ldots, x_N) \right),
$$
  with mean $m$ and covariance $K$.
\end{frame}

\begin{frame}[plain,fragile]
\frametitle{Topics}

\begin{itemize}
\item More matematical details

\item The role of the covariance function (different kernels)

\item multidimensional case

\item examples.
\end{itemize}

\noindent
\end{frame}

\end{document}
