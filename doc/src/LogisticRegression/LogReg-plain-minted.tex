%%
%% Automatically generated file from DocOnce source
%% (https://github.com/hplgit/doconce/)
%%
%%


%-------------------- begin preamble ----------------------

\documentclass[%
oneside,                 % oneside: electronic viewing, twoside: printing
final,                   % draft: marks overfull hboxes, figures with paths
10pt]{article}

\listfiles               %  print all files needed to compile this document

\usepackage{relsize,makeidx,color,setspace,amsmath,amsfonts,amssymb}
\usepackage[table]{xcolor}
\usepackage{bm,ltablex,microtype}

\usepackage[pdftex]{graphicx}

\usepackage{fancyvrb} % packages needed for verbatim environments
\usepackage{minted}
\usemintedstyle{default}

\usepackage[T1]{fontenc}
%\usepackage[latin1]{inputenc}
\usepackage{ucs}
\usepackage[utf8x]{inputenc}

\usepackage{lmodern}         % Latin Modern fonts derived from Computer Modern

% Hyperlinks in PDF:
\definecolor{linkcolor}{rgb}{0,0,0.4}
\usepackage{hyperref}
\hypersetup{
    breaklinks=true,
    colorlinks=true,
    linkcolor=linkcolor,
    urlcolor=linkcolor,
    citecolor=black,
    filecolor=black,
    %filecolor=blue,
    pdfmenubar=true,
    pdftoolbar=true,
    bookmarksdepth=3   % Uncomment (and tweak) for PDF bookmarks with more levels than the TOC
    }
%\hyperbaseurl{}   % hyperlinks are relative to this root

\setcounter{tocdepth}{2}  % levels in table of contents

% --- fancyhdr package for fancy headers ---
\usepackage{fancyhdr}
\fancyhf{} % sets both header and footer to nothing
\renewcommand{\headrulewidth}{0pt}
\fancyfoot[LE,RO]{\thepage}
% Ensure copyright on titlepage (article style) and chapter pages (book style)
\fancypagestyle{plain}{
  \fancyhf{}
  \fancyfoot[C]{{\footnotesize \copyright\ 1999-2019, Morten Hjorth-Jensen. Released under CC Attribution-NonCommercial 4.0 license}}
%  \renewcommand{\footrulewidth}{0mm}
  \renewcommand{\headrulewidth}{0mm}
}
% Ensure copyright on titlepages with \thispagestyle{empty}
\fancypagestyle{empty}{
  \fancyhf{}
  \fancyfoot[C]{{\footnotesize \copyright\ 1999-2019, Morten Hjorth-Jensen. Released under CC Attribution-NonCommercial 4.0 license}}
  \renewcommand{\footrulewidth}{0mm}
  \renewcommand{\headrulewidth}{0mm}
}

\pagestyle{fancy}


% prevent orhpans and widows
\clubpenalty = 10000
\widowpenalty = 10000

% --- end of standard preamble for documents ---


% insert custom LaTeX commands...

\raggedbottom
\makeindex
\usepackage[totoc]{idxlayout}   % for index in the toc
\usepackage[nottoc]{tocbibind}  % for references/bibliography in the toc

%-------------------- end preamble ----------------------

\begin{document}

% matching end for #ifdef PREAMBLE

\newcommand{\exercisesection}[1]{\subsection*{#1}}


% ------------------- main content ----------------------



% ----------------- title -------------------------

\thispagestyle{empty}

\begin{center}
{\LARGE\bf
\begin{spacing}{1.25}
Data Analysis and Machine Learning: Logistic Regression
\end{spacing}
}
\end{center}

% ----------------- author(s) -------------------------

\begin{center}
{\bf Morten Hjorth-Jensen${}^{1, 2}$} \\ [0mm]
\end{center}

\begin{center}
% List of all institutions:
\centerline{{\small ${}^1$Department of Physics, University of Oslo}}
\centerline{{\small ${}^2$Department of Physics and Astronomy and National Superconducting Cyclotron Laboratory, Michigan State University}}
\end{center}
    
% ----------------- end author(s) -------------------------

% --- begin date ---
\begin{center}
Sep 19, 2019
\end{center}
% --- end date ---

\vspace{1cm}


% !split 
\subsection*{Logistic Regression}

In linear regression our main interest was centered on learning the
coefficients of a functional fit (say a polynomial) in order to be
able to predict the response of a continuous variable on some unseen
data. The fit to the continuous variable $y_i$ is based on some
independent variables $\hat{x}_i$. Linear regression resulted in
analytical expressions for standard ordinary Least Squares or Ridge
regression (in terms of matrices to invert) for several quantities,
ranging from the variance and thereby the confidence intervals of the
parameters $\hat{\beta}$ to the mean squared error. If we can invert
the product of the design matrices, linear regression gives then a
simple recipe for fitting our data.


Classification problems, however, are concerned with outcomes taking
the form of discrete variables (i.e.~categories). We may for example,
on the basis of DNA sequencing for a number of patients, like to find
out which mutations are important for a certain disease; or based on
scans of various patients' brains, figure out if there is a tumor or
not; or given a specific physical system, we'd like to identify its
state, say whether it is an ordered or disordered system (typical
situation in solid state physics); or classify the status of a
patient, whether she/he has a stroke or not and many other similar
situations.

The most common situation we encounter when we apply logistic
regression is that of two possible outcomes, normally denoted as a
binary outcome, true or false, positive or negative, success or
failure etc.

% !split
\subsection*{Optimization and Deep learning}

Logistic regression will also serve as our stepping stone towards neural
network algorithms and supervised deep learning. For logistic
learning, the minimization of the cost function leads to a non-linear
equation in the parameters $\hat{\beta}$. The optimization of the problem calls therefore for minimization algorithms. This forms the bottle neck of all machine learning algorithms, namely how to find reliable minima of a multi-variable function. This leads us to the family of gradient descent methods. The latter are the working horses of basically all modern machine learning algorithms. 

We note also that many of the topics discussed here 
regression are also commonly used in modern supervised Deep Learning
models, as we will see later.


% !split 
\subsection*{Basics}

We consider the case where the dependent variables, also called the
responses or the outcomes, $y_i$ are discrete and only take values
from $k=0,\dots,K-1$ (i.e.~$K$ classes).

The goal is to predict the
output classes from the design matrix $\hat{X}\in\mathbb{R}^{n\times p}$
made of $n$ samples, each of which carries $p$ features or predictors. The
primary goal is to identify the classes to which new unseen samples
belong.

Let us specialize to the case of two classes only, with outputs
$y_i=0$ and $y_i=1$. Our outcomes could represent the status of a
credit card user that could default or not on her/his credit card
debt. That is


\[
y_i = \begin{bmatrix} 0 & \mathrm{no}\\  1 & \mathrm{yes} \end{bmatrix}.
\]



% !split
\subsection*{Linear classifier}

Before moving to the logistic model, let us try to use our linear regression model to classify these two outcomes. We could for example fit a linear model to the default case if $y_i > 0.5$ and the no default case $y_i \leq 0.5$. 

We would then have our 
weighted linear combination, namely 
\begin{equation}
\hat{y} = \hat{X}^T\hat{\beta} +  \hat{\epsilon},
\end{equation}
where $\hat{y}$ is a vector representing the possible outcomes, $\hat{X}$ is our
$n\times p$ design matrix and $\hat{\beta}$ represents our estimators/predictors.

% !split
\subsection*{Some selected properties}

The main problem with our  function is that it 
takes values on the entire real axis. In the case of
logistic regression, however, the labels $y_i$ are discrete
variables. A typical example is the credit card data discussed below here, where we can set the state of defaulting the debt to $y_i=1$ and not to $y_i=0$ for one the persons in the data set (see the full example below).

One simple way to get a discrete output is to have sign
functions that map the output of a linear regressor to values $\{0,1\}$,
$f(s_i)=sign(s_i)=1$ if $s_i\ge 0$ and 0 if otherwise. 
We will encounter this model in our first demonstration of neural networks. Historically it is called the ``perceptron" model in the machine learning
literature. This model is extremely simple. However, in many cases it is more
favorable to use a ``soft" classifier that outputs
the probability of a given category. This leads us to the logistic function.


% !split
\subsection*{The logistic function}

The perceptron is an example of a ``hard classification'' model. We
will encounter this model when we discuss neural networks as
well. Each datapoint is deterministically assigned to a category (i.e
$y_i=0$ or $y_i=1$). In many cases, it is favorable to have a ``soft''
classifier that outputs the probability of a given category rather
than a single value. For example, given $x_i$, the classifier
outputs the probability of being in a category $k$.  Logistic regression
is the most common example of a so-called soft classifier. In logistic
regression, the probability that a data point $x_i$
belongs to a category $y_i=\{0,1\}$ is given by the so-called logit function (or Sigmoid) which is meant to represent the likelihood for a given event, 
\[
p(t) = \frac{1}{1+\mathrm \exp{-t}}=\frac{\exp{t}}{1+\mathrm \exp{t}}.
\]
Note that $1-p(t)= p(-t)$.
The following code plots the logistic function, the step function and other functions we will encounter from here and on.


\begin{minted}[fontsize=\fontsize{9pt}{9pt},linenos=false,mathescape,baselinestretch=1.0,fontfamily=tt,xleftmargin=7mm]{python}
"""The sigmoid function (or the logistic curve) is a
function that takes any real number, z, and outputs a number (0,1).
It is useful in neural networks for assigning weights on a relative scale.
The value z is the weighted sum of parameters involved in the learning algorithm."""

import numpy
import matplotlib.pyplot as plt
import math as mt

z = numpy.arange(-5, 5, .1)
sigma_fn = numpy.vectorize(lambda z: 1/(1+numpy.exp(-z)))
sigma = sigma_fn(z)

fig = plt.figure()
ax = fig.add_subplot(111)
ax.plot(z, sigma)
ax.set_ylim([-0.1, 1.1])
ax.set_xlim([-5,5])
ax.grid(True)
ax.set_xlabel('z')
ax.set_title('sigmoid function')

plt.show()

"""Step Function"""
z = numpy.arange(-5, 5, .02)
step_fn = numpy.vectorize(lambda z: 1.0 if z >= 0.0 else 0.0)
step = step_fn(z)

fig = plt.figure()
ax = fig.add_subplot(111)
ax.plot(z, step)
ax.set_ylim([-0.5, 1.5])
ax.set_xlim([-5,5])
ax.grid(True)
ax.set_xlabel('z')
ax.set_title('step function')

plt.show()

"""tanh Function"""
z = numpy.arange(-2*mt.pi, 2*mt.pi, 0.1)
t = numpy.tanh(z)

fig = plt.figure()
ax = fig.add_subplot(111)
ax.plot(z, t)
ax.set_ylim([-1.0, 1.0])
ax.set_xlim([-2*mt.pi,2*mt.pi])
ax.grid(True)
ax.set_xlabel('z')
ax.set_title('tanh function')

plt.show()
\end{minted}







% !split
\subsection*{Two parameters}

We assume now that we have two classes with $y_i$ either $0$ or $1$. Furthermore we assume also that we have only two parameters $\beta$ in our fitting of the Sigmoid function, that is we define probabilities 
\begin{align*}
p(y_i=1|x_i,\hat{\beta}) &= \frac{\exp{(\beta_0+\beta_1x_i)}}{1+\exp{(\beta_0+\beta_1x_i)}},\nonumber\\
p(y_i=0|x_i,\hat{\beta}) &= 1 - p(y_i=1|x_i,\hat{\beta}),
\end{align*}
where $\hat{\beta}$ are the weights we wish to extract from data, in our case $\beta_0$ and $\beta_1$. 

Note that we used
\[
p(y_i=0\vert x_i, \hat{\beta}) = 1-p(y_i=1\vert x_i, \hat{\beta}).
\]

% !split 
\subsection*{Maximum likelihood}

In order to define the total likelihood for all possible outcomes from a  
dataset $\mathcal{D}=\{(y_i,x_i)\}$, with the binary labels
$y_i\in\{0,1\}$ and where the data points are drawn independently, we use the so-called \href{{https://en.wikipedia.org/wiki/Maximum_likelihood_estimation}}{Maximum Likelihood Estimation} (MLE) principle. 
We aim thus at maximizing 
the probability of seeing the observed data. We can then approximate the 
likelihood in terms of the product of the individual probabilities of a specific outcome $y_i$, that is 
\begin{align*}
P(\mathcal{D}|\hat{\beta})& = \prod_{i=1}^n \left[p(y_i=1|x_i,\hat{\beta})\right]^{y_i}\left[1-p(y_i=1|x_i,\hat{\beta}))\right]^{1-y_i}\nonumber \\
\end{align*}
from which we obtain the log-likelihood and our \textbf{cost/loss} function
\[
\mathcal{C}(\hat{\beta}) = \sum_{i=1}^n \left( y_i\log{p(y_i=1|x_i,\hat{\beta})} + (1-y_i)\log\left[1-p(y_i=1|x_i,\hat{\beta}))\right]\right).
\]

% !split
\subsection*{The cost function rewritten}

Reordering the logarithms, we can rewrite the \textbf{cost/loss} function as
\[
\mathcal{C}(\hat{\beta}) = \sum_{i=1}^n  \left(y_i(\beta_0+\beta_1x_i) -\log{(1+\exp{(\beta_0+\beta_1x_i)})}\right).
\]

The maximum likelihood estimator is defined as the set of parameters that maximize the log-likelihood where we maximize with respect to $\beta$.
Since the cost (error) function is just the negative log-likelihood, for logistic regression we have that
\[
\mathcal{C}(\hat{\beta})=-\sum_{i=1}^n  \left(y_i(\beta_0+\beta_1x_i) -\log{(1+\exp{(\beta_0+\beta_1x_i)})}\right).
\]
This equation is known in statistics as the \textbf{cross entropy}. Finally, we note that just as in linear regression, 
in practice we often supplement the cross-entropy with additional regularization terms, usually $L_1$ and $L_2$ regularization as we did for Ridge and Lasso regression.

% !split
\subsection*{Minimizing the cross entropy}

The cross entropy is a convex function of the weights $\hat{\beta}$ and,
therefore, any local minimizer is a global minimizer. 


Minimizing this
cost function with respect to the two parameters $\beta_0$ and $\beta_1$ we obtain

\[
\frac{\partial \mathcal{C}(\hat{\beta})}{\partial \beta_0} = -\sum_{i=1}^n  \left(y_i -\frac{\exp{(\beta_0+\beta_1x_i)}}{1+\exp{(\beta_0+\beta_1x_i)}}\right),
\]
and 
\[
\frac{\partial \mathcal{C}(\hat{\beta})}{\partial \beta_1} = -\sum_{i=1}^n  \left(y_ix_i -x_i\frac{\exp{(\beta_0+\beta_1x_i)}}{1+\exp{(\beta_0+\beta_1x_i)}}\right).
\]

% !split
\subsection*{A more compact expression}

Let us now define a vector $\hat{y}$ with $n$ elements $y_i$, an
$n\times p$ matrix $\hat{X}$ which contains the $x_i$ values and a
vector $\hat{p}$ of fitted probabilities $p(y_i\vert x_i,\hat{\beta})$. We can rewrite in a more compact form the first
derivative of cost function as

\[
\frac{\partial \mathcal{C}(\hat{\beta})}{\partial \hat{\beta}} = -\hat{X}^T\left(\hat{y}-\hat{p}\right). 
\]

If we in addition define a diagonal matrix $\hat{W}$ with elements 
$p(y_i\vert x_i,\hat{\beta})(1-p(y_i\vert x_i,\hat{\beta})$, we can obtain a compact expression of the second derivative as 

\[
\frac{\partial^2 \mathcal{C}(\hat{\beta})}{\partial \hat{\beta}\partial \hat{\beta}^T} = \hat{X}^T\hat{W}\hat{X}. 
\]

% !split
\subsection*{Extending to more predictors}

Within a binary classification problem, we can easily expand our model to include multiple predictors. Our ratio between likelihoods is then with $p$ predictors
\[
\log{ \frac{p(\hat{\beta}\hat{x})}{1-p(\hat{\beta}\hat{x})}} = \beta_0+\beta_1x_1+\beta_2x_2+\dots+\beta_px_p.
\]
Here we defined $\hat{x}=[1,x_1,x_2,\dots,x_p]$ and $\hat{\beta}=[\beta_0, \beta_1, \dots, \beta_p]$ leading to
\[
p(\hat{\beta}\hat{x})=\frac{ \exp{(\beta_0+\beta_1x_1+\beta_2x_2+\dots+\beta_px_p)}}{1+\exp{(\beta_0+\beta_1x_1+\beta_2x_2+\dots+\beta_px_p)}}.
\]

% !split
\subsection*{Including more classes}

Till now we have mainly focused on two classes, the so-called binary
system. Suppose we wish to extend to $K$ classes.  Let us for the sake
of simplicity assume we have only two predictors. We have then
following model

\[
\log{\frac{p(C=1\vert x)}{p(K\vert x)}} = \beta_{10}+\beta_{11}x_1,
\]
\[
\log{\frac{p(C=2\vert x)}{p(K\vert x)}} = \beta_{20}+\beta_{21}x_1,
\]
and so on till the class $C=K-1$ class
\[
\log{\frac{p(C=K-1\vert x)}{p(K\vert x)}} = \beta_{(K-1)0}+\beta_{(K-1)1}x_1,
\]

and the model is specified in term of $K-1$ so-called log-odds or
\textbf{logit} transformations.


% !split
\subsection*{The Softmax function}

In our discussion of neural networks we will encounter the above again
in terms of the so-called \textbf{Softmax} function.

The softmax function is used in various multiclass classification
methods, such as multinomial logistic regression (also known as
softmax regression), multiclass linear discriminant analysis, naive
Bayes classifiers, and artificial neural networks.  Specifically, in
multinomial logistic regression and linear discriminant analysis, the
input to the function is the result of $K$ distinct linear functions,
and the predicted probability for the $k$-th class given a sample
vector $\hat{x}$ and a weighting vector $\hat{\beta}$ is (with two
predictors):

\[
p(C=k\vert \mathbf {x} )=\frac{\exp{(\beta_{k0}+\beta_{k1}x_1)}}{1+\sum_{l=1}^{K-1}\exp{(\beta_{l0}+\beta_{l1}x_1)}}.
\]
It is easy to extend to more predictors. The final class is 
\[
p(C=K\vert \mathbf {x} )=\frac{1}{1+\sum_{l=1}^{K-1}\exp{(\beta_{l0}+\beta_{l1}x_1)}},
\]

and they sum to one. Our earlier discussions were all specialized to
the case with two classes only. It is easy to see from the above that
what we derived earlier is compatible with these equations.

To find the optimal parameters we would typically use a gradient
descent method.  Newton's method and gradient descent methods are
discussed in the material on \href{{https://compphysics.github.io/MachineLearning/doc/pub/Splines/html/Splines-bs.html}}{optimization
methods}.




% !split
\subsection*{A simple classification problem}
\begin{minted}[fontsize=\fontsize{9pt}{9pt},linenos=false,mathescape,baselinestretch=1.0,fontfamily=tt,xleftmargin=7mm]{python}
import numpy as np
from sklearn import datasets, linear_model
import matplotlib.pyplot as plt


def generate_data():
    np.random.seed(0)
    X, y = datasets.make_moons(200, noise=0.20)
    return X, y


def visualize(X, y, clf):
    plot_decision_boundary(lambda x: clf.predict(x), X, y)

def plot_decision_boundary(pred_func, X, y):
    # Set min and max values and give it some padding
    x_min, x_max = X[:, 0].min() - .5, X[:, 0].max() + .5
    y_min, y_max = X[:, 1].min() - .5, X[:, 1].max() + .5
    h = 0.01
    # Generate a grid of points with distance h between them
    xx, yy = np.meshgrid(np.arange(x_min, x_max, h), np.arange(y_min, y_max, h))
    # Predict the function value for the whole gid
    Z = pred_func(np.c_[xx.ravel(), yy.ravel()])
    Z = Z.reshape(xx.shape)
    # Plot the contour and training examples
    plt.contourf(xx, yy, Z, cmap=plt.cm.Spectral)
    plt.scatter(X[:, 0], X[:, 1], c=y, cmap=plt.cm.Spectral)
    plt.show()


def classify(X, y):
    clf = linear_model.LogisticRegressionCV()
    clf.fit(X, y)
    return clf


def main():
    X, y = generate_data()
    # visualize(X, y)
    clf = classify(X, y)
    visualize(X, y, clf)

if __name__ == "__main__":
    main()
\end{minted}



% !split
\subsection*{The Credit Card example}
Here we use the the \href{{https://archive.ics.uci.edu/ml/datasets/default+of+credit+card+clients}}{credit card data}. More text to come.

\begin{minted}[fontsize=\fontsize{9pt}{9pt},linenos=false,mathescape,baselinestretch=1.0,fontfamily=tt,xleftmargin=7mm]{python}
import pandas as pd
import os
import numpy as np


from sklearn.model_selection import train_test_split
from sklearn.preprocessing import OneHotEncoder
from sklearn.compose import ColumnTransformer
from sklearn.preprocessing import StandardScaler, OneHotEncoder
from sklearn.metrics import confusion_matrix, accuracy_score, roc_auc_score

# Trying to set the seed
np.random.seed(0)
import random
random.seed(0)

# Reading file into data frame
cwd = os.getcwd()
filename = cwd + '/default of credit card clients.xls'
nanDict = {}
df = pd.read_excel(filename, header=1, skiprows=0, index_col=0, na_values=nanDict)

df.rename(index=str, columns={"default payment next month": "defaultPaymentNextMonth"}, inplace=True)

# Features and targets 
X = df.loc[:, df.columns != 'defaultPaymentNextMonth'].values
y = df.loc[:, df.columns == 'defaultPaymentNextMonth'].values

# Categorical variables to one-hot's
onehotencoder = OneHotEncoder(categories="auto")

X = ColumnTransformer(
    [("", onehotencoder, [3]),],
    remainder="passthrough"
).fit_transform(X)

y.shape

# Train-test split
trainingShare = 0.5 
seed  = 1
XTrain, XTest, yTrain, yTest=train_test_split(X, y, train_size=trainingShare, \
                                              test_size = 1-trainingShare,
                                             random_state=seed)

# Input Scaling
sc = StandardScaler()
XTrain = sc.fit_transform(XTrain)
XTest = sc.transform(XTest)

# One-hot's of the target vector
Y_train_onehot, Y_test_onehot = onehotencoder.fit_transform(yTrain), onehotencoder.fit_transform(yTest)

# Remove instances with zeros only for past bill statements or paid amounts
'''
df = df.drop(df[(df.BILL_AMT1 == 0) &
                (df.BILL_AMT2 == 0) &
                (df.BILL_AMT3 == 0) &
                (df.BILL_AMT4 == 0) &
                (df.BILL_AMT5 == 0) &
                (df.BILL_AMT6 == 0) &
                (df.PAY_AMT1 == 0) &
                (df.PAY_AMT2 == 0) &
                (df.PAY_AMT3 == 0) &
                (df.PAY_AMT4 == 0) &
                (df.PAY_AMT5 == 0) &
                (df.PAY_AMT6 == 0)].index)
'''
df = df.drop(df[(df.BILL_AMT1 == 0) &
                (df.BILL_AMT2 == 0) &
                (df.BILL_AMT3 == 0) &
                (df.BILL_AMT4 == 0) &
                (df.BILL_AMT5 == 0) &
                (df.BILL_AMT6 == 0)].index)

df = df.drop(df[(df.PAY_AMT1 == 0) &
                (df.PAY_AMT2 == 0) &
                (df.PAY_AMT3 == 0) &
                (df.PAY_AMT4 == 0) &
                (df.PAY_AMT5 == 0) &
                (df.PAY_AMT6 == 0)].index)

from sklearn.linear_model import LogisticRegression
from sklearn.model_selection import GridSearchCV

lambdas=np.logspace(-5,7,13)
parameters = [{'C': 1./lambdas, "solver":["lbfgs"]}]#*len(parameters)}]
scoring = ['accuracy', 'roc_auc']
logReg = LogisticRegression()
gridSearch = GridSearchCV(logReg, parameters, cv=5, scoring=scoring, refit='roc_auc') 
\end{minted}

\begin{minted}[fontsize=\fontsize{9pt}{9pt},linenos=false,mathescape,baselinestretch=1.0,fontfamily=tt,xleftmargin=7mm]{python}

# "refit" gives the metric used deciding best model. 
# See more http://scikit-learn.org/stable/auto_examples/model_selection/plot_multi_metric_evaluation.html
gridSearch.fit(XTrain, yTrain.ravel())

def gridSearchSummary(method, scoring):
    """Prints best parameters from Grid search
    and AUC with standard deviation for all 
    parameter combos """
    
    method = eval(method)
    if scoring == 'accuracy':
        mean = 'mean_test_score'
        sd = 'std_test_score'
    elif scoring == 'auc':
        mean = 'mean_test_roc_auc'
        sd = 'std_test_roc_auc'
    print("Best: %f using %s" % (method.best_score_, method.best_params_))
    means = method.cv_results_[mean]
    stds = method.cv_results_[sd]
    params = method.cv_results_['params']
    for mean, stdev, param in zip(means, stds, params):
        print("%f (%f) with: %r" % (mean, stdev, param))

def createConfusionMatrix(method, printOut=True):
    """
    Computes and prints confusion matrices, accuracy scores,
    and AUC for test and training sets 
    """
    confusionArray = np.zeros(6, dtype=object)
    method = eval(method)
    
    # Train
    yPredTrain = method.predict(XTrain)
    yPredTrain = (yPredTrain > 0.5)
    cm = confusion_matrix(
        yTrain, yPredTrain) 
    cm = np.around(cm/cm.sum(axis=1)[:,None], 2)
    confusionArray[0] = cm
    
    accScore = accuracy_score(yTrain, yPredTrain)
    confusionArray[1] = accScore
    
    AUC = roc_auc_score(yTrain, yPredTrain)
    confusionArray[2] = AUC
    
    if printOut:
        print('\n###################  Training  ###############')
        print('\nTraining Confusion matrix: \n', cm)
        print('\nTraining Accuracy score: \n', accScore)
        print('\nTrain AUC: \n', AUC)
    
    # Test
    yPred = method.predict(XTest)
    yPred = (yPred > 0.5)
    cm = confusion_matrix(
        yTest, yPred) 
    cm = np.around(cm/cm.sum(axis=1)[:,None], 2)
    confusionArray[3] = cm
    
    accScore = accuracy_score(yTest, yPred)
    confusionArray[4] = accScore
    
    AUC = roc_auc_score(yTest, yPred)
    confusionArray[5] = AUC
    
    if printOut:
        print('\n###################  Testing  ###############')
        print('\nTest Confusion matrix: \n', cm)
        print('\nTest Accuracy score: \n', accScore)
        print('\nTestAUC: \n', AUC)    
    
    return confusionArray


import matplotlib.pyplot as plt
import seaborn
import scikitplot as skplt

seaborn.set(style="white", context="notebook", font_scale=1.5, 
            rc={"axes.grid": True, "legend.frameon": False,
"lines.markeredgewidth": 1.4, "lines.markersize": 10})
seaborn.set_context("notebook", font_scale=1.5, rc={"lines.linewidth": 4.5})

yPred = gridSearch.predict_proba(XTest) 
print(yTest.ravel().shape, yPred.shape)

#skplt.metrics.plot_cumulative_gain(yTest.ravel(), yPred_onehot)
skplt.metrics.plot_cumulative_gain(yTest.ravel(), yPred)

defaults = sum(yTest == 1)
total = len(yTest)
defaultRate = defaults/total
def bestCurve(defaults, total, defaultRate):
    x = np.linspace(0, 1, total)
    
    y1 = np.linspace(0, 1, defaults)
    y2 = np.ones(total-defaults)
    y3 = np.concatenate([y1,y2])
    return x, y3

x, best = bestCurve(defaults=defaults, total=total, defaultRate=defaultRate)    
plt.plot(x, best)    


plt.show()

\end{minted}

% ------------------- end of main content ---------------

\end{document}

