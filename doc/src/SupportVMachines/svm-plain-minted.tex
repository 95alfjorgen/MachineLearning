%%
%% Automatically generated file from DocOnce source
%% (https://github.com/hplgit/doconce/)
%%
%%


%-------------------- begin preamble ----------------------

\documentclass[%
oneside,                 % oneside: electronic viewing, twoside: printing
final,                   % draft: marks overfull hboxes, figures with paths
10pt]{article}

\listfiles               %  print all files needed to compile this document

\usepackage{relsize,makeidx,color,setspace,amsmath,amsfonts,amssymb}
\usepackage[table]{xcolor}
\usepackage{bm,ltablex,microtype}

\usepackage[pdftex]{graphicx}

\usepackage{fancyvrb} % packages needed for verbatim environments
\usepackage{minted}
\usemintedstyle{default}

\usepackage[T1]{fontenc}
%\usepackage[latin1]{inputenc}
\usepackage{ucs}
\usepackage[utf8x]{inputenc}

\usepackage{lmodern}         % Latin Modern fonts derived from Computer Modern

% Hyperlinks in PDF:
\definecolor{linkcolor}{rgb}{0,0,0.4}
\usepackage{hyperref}
\hypersetup{
    breaklinks=true,
    colorlinks=true,
    linkcolor=linkcolor,
    urlcolor=linkcolor,
    citecolor=black,
    filecolor=black,
    %filecolor=blue,
    pdfmenubar=true,
    pdftoolbar=true,
    bookmarksdepth=3   % Uncomment (and tweak) for PDF bookmarks with more levels than the TOC
    }
%\hyperbaseurl{}   % hyperlinks are relative to this root

\setcounter{tocdepth}{2}  % levels in table of contents

% --- fancyhdr package for fancy headers ---
\usepackage{fancyhdr}
\fancyhf{} % sets both header and footer to nothing
\renewcommand{\headrulewidth}{0pt}
\fancyfoot[LE,RO]{\thepage}
% Ensure copyright on titlepage (article style) and chapter pages (book style)
\fancypagestyle{plain}{
  \fancyhf{}
  \fancyfoot[C]{{\footnotesize \copyright\ 1999-2018, Morten Hjorth-Jensen. Released under CC Attribution-NonCommercial 4.0 license}}
%  \renewcommand{\footrulewidth}{0mm}
  \renewcommand{\headrulewidth}{0mm}
}
% Ensure copyright on titlepages with \thispagestyle{empty}
\fancypagestyle{empty}{
  \fancyhf{}
  \fancyfoot[C]{{\footnotesize \copyright\ 1999-2018, Morten Hjorth-Jensen. Released under CC Attribution-NonCommercial 4.0 license}}
  \renewcommand{\footrulewidth}{0mm}
  \renewcommand{\headrulewidth}{0mm}
}

\pagestyle{fancy}


% prevent orhpans and widows
\clubpenalty = 10000
\widowpenalty = 10000

% --- end of standard preamble for documents ---


% insert custom LaTeX commands...

\raggedbottom
\makeindex
\usepackage[totoc]{idxlayout}   % for index in the toc
\usepackage[nottoc]{tocbibind}  % for references/bibliography in the toc

%-------------------- end preamble ----------------------

\begin{document}

% matching end for #ifdef PREAMBLE

\newcommand{\exercisesection}[1]{\subsection*{#1}}


% ------------------- main content ----------------------



% ----------------- title -------------------------

\thispagestyle{empty}

\begin{center}
{\LARGE\bf
\begin{spacing}{1.25}
Data Analysis and Machine Learning: Support Vector Machines
\end{spacing}
}
\end{center}

% ----------------- author(s) -------------------------

\begin{center}
{\bf Morten Hjorth-Jensen${}^{1, 2}$} \\ [0mm]
\end{center}

\begin{center}
% List of all institutions:
\centerline{{\small ${}^1$Department of Physics, University of Oslo}}
\centerline{{\small ${}^2$Department of Physics and Astronomy and National Superconducting Cyclotron Laboratory, Michigan State University}}
\end{center}
    
% ----------------- end author(s) -------------------------

% --- begin date ---
\begin{center}
Nov 5, 2018
\end{center}
% --- end date ---

\vspace{1cm}


% !split
\subsection*{Support Vector Machines, overarching aims}

A Support Vector Machine (SVM) is a very powerful and versatile
Machine Learning model, capable of performing linear or nonlinear
classification, regression, and even outlier detection. It is one of
the most popular models in Machine Learning, and anyone interested in
Machine Learning should have it in their toolbox. SVMs are
particularly well suited for classification of complex but small-sized or
medium-sized datasets.  

The case with two well-separated classes only can be understood in an intuitive way in terms of lines in a two-dimensional space separating the two classes (see figure below).  

The basic mathematics behind the SVM is however less familiar to most of us. 
It relies on the definition of hyperplanes and the
definition of a \textbf{margin} which separates classes (in case of
classification problems) of variables. It is also used for regression
problems.

With SVMs we distinguish between hard margin and soft margins. The latter introduces a so-called softening parameter to be discussed below.
We distinguish also between linear and non-linear approaches. The latter are the most frequent ones since it is rather unlikely that we can separate classes easily by say straight lines. 




% !split
\subsection*{Hyperplanes and all that}

The theory behind support vector machines (SVM hereafter) is based on
the mathematical description of so-called hyperplanes. Let us start
with a two-dimensional case. This will also allow us to introduce our
first SVM examples. These will be tailored to the case of two specific
classes, as displayed in the figure here.

We assume here that our data set can be well separated into two
domains, where a straight line does the job in the separating the two
classes. Here the two classes are represented by either crosses or
circles.

% !split
\subsection*{What is a hyperplane?}

The aim of the SVM algorithm is to find a hyperplane in an $p$-dimensional space, where $p$ is the number of features  that distinctly classifies the data points.  

In a $p$-dimensional space, a hyperplane is what we call an affine subspace of dimension of $p-1$.
As an example, in two dimension, a hyperplane is simply as straight line while in three dimensions it is 
a two-dimensional subspace, or stated simply, a plane. 

In two dimensions, with the variables $x_1$ and $x_2$, the hyperplane is defined as
\[
b+w_1x_1+w_2x_2=0,
\]
where $b$ is the intercept and $w_1$ and $w_2$ define the elements of a vector orthogonal to the line 
$b+w_1x_1+w_2x_2=0$. 
In two dimensions we define the vectors $\hat{x} =[x1,x2]$ and $\hat{w}=[w1,w2]$. 
We can then rewrite the above equation as 
\[
\hat{w}^T\hat{x}+b=0.
\]

% !split
\subsection*{A $p$-dimensional space of features}

We limit ourselves to two classes of outputs $y_i$ and assign these classes the values $y_i = \pm 1$. 
In a $p$-dimensional space of say $p$ features we have a hyperplane defines as 
\[
b+wx_1+w_2x_2+\dots +w_px_p=0.
\]
If we define a 
matrix $\hat{X}=\left[\hat{x}_1,\hat{x}_2,\dots, \hat{x}_p\right]$
of dimension $n\times p$, where $n$ represents the observations for each feature and each vector $x_i$ is a column vector of the matrix $\hat{X}$, 
\[
\hat{x}_i = \begin{bmatrix} x_{i1} \\ x_{i2} \\ \dots \\ \dots \\ x_{ip} \end{bmatrix}.
\]
If the above condition is not met for a given vector $\hat{x}_i$ we have 
\[
b+w_1x_{i1}+w_2x_{i}2+\dots +w_px_{ip} >0,
\]
if our output $y_i=1$.
In this case we say that $\hat{x}_i$ lies on one of the sides of the hyperplane and if 
\[
b+w_1x_{i1}+w_2x_{i}2+\dots +w_px_{ip} < 0,
\]
for the class of observations $y_i=-1$, 
then $\hat{x}_i$ lies on the other side. 

Equivalently, for the two classes of observations we have 
\[
y_i\left(b+w_1x_{i1}+w_2x_{i}2+\dots +w_px_{ip}\right) > 0. 
\]

When we try to separate hyperplanes, if it exists, we can use it to construct a natural classifier: a test observation is assigned a given class depending on which side of the hyperplane it is located.

% !split 
\subsection*{The two-dimensional case}

Let us try to develop our intuition about SVMs by limiting ourselves to a two-dimensional
plane.  To separate the two classes of data points, there are many
possible lines (hyperplanes if you prefer a more strict naming)  
that could be chosen. Our objective is to find a
plane that has the maximum margin, i.e the maximum distance between
data points of both classes. Maximizing the margin distance provides
some reinforcement so that future data points can be classified with
more confidence.

What a linear classifier attempts to accomplish is to split the
feature space into two half spaces by placing a hyperplane between the
data points.  This hyperplane will be our decision boundary.  All
points on one side of the plane will belong to class one and all points
on the other side of the plane will belong to the second class two.

Unfortunately there are many ways in which we can place a hyperplane
to divide the data.  Below is an example of two candidate hyperplanes
for our data sample.

% !split
\subsection*{Getting into the details}

Let us define the function
\[
f(x) = \hat{w}^T\hat{x}+b = 0,
\]
as the function that determines the line $L$ that separates two classes (our two features), see the figure here. 


Any point defined by $\hat{x}_i$ and $\hat{x}_2$ on the line $L$ will satisfy $\hat{w}^T(\hat{x}_1-\hat{x}_2)=0$. 

The signed distance $\delta$ from any point defined by a vector $\hat{x}$ and a point $\hat{x}_0$ on the line $L$ is then
\[
\delta = \frac{1}{\vert\vert \hat{w}\vert\vert}(\hat{w}^T\hat{x}+b).
\]

% !split
\subsection*{First attempt at a minimization approach}

How do we find the parameter $b$ and the vector $\hat{w}$? What we could
do is to define a cost function which now contains the set of all
misclassified points $M$ and attempt to minimize this function

\[
C(\hat{w},b) = -\sum_{i\in M} y_i(\hat{w}^T\hat{x}_i+b).
\]

We could now for example define all values $y_i =1$ as misclassified in case we have $\hat{w}^T\hat{x}_i+b < 0$ and the opposite if we have $y_i=-1$. Taking the derivatives gives us
\[
\frac{\partial C}{\partial b} = -\sum_{i\in M} y_i,
\]
and 
\[
\frac{\partial C}{\partial \hat{w}} = -\sum_{i\in M} y_ix_i.
\]

% !split
\subsection*{Solving the equations}

We can now use the Newton-Raphson method or gradient descent to solve the equations
\[
b \leftarrow b +\eta \frac{\partial C}{\partial b},
\]
and
\[
\hat{w} \leftarrow \hat{w} +\eta \frac{\partial C}{\partial \hat{w}},
\]
where $\eta$ is our by now well-known learning rate. 

There are however problems with this approach, although it looks
pretty straightforward to implement. In case we separate our data into
two distinct classes, we may up with many possible lines, as indicated
in the figure and shown by running the following program. For small
gaps between the entries, we may also end up needing many iterations
before the solutions converge and if the data cannot be separated
properly into two distinct classes, we may not experience a converge
at all.

% !split
\subsection*{A better approach}

A better approach is rather to try to define a large margin between
the two classes (if they are well separated from the beginning).

Thus, we wish to find a margin $M$ with $\hat{w}$ normalized to
$\vert\vert \hat{w}\vert\vert =1$ subject to the condition

\[
y_i(\hat{w}^T\hat{x}_i+b) \geq M \forall i=1,2,\dots, p. 
\]
All points are thus at a signed distance from the decision boundary defined by the line $L$. The parameters $b$ and $w_1$ and $w_2$ define this line. 

We seek thus the largest value $M$ defined by
\[
\frac{1}{\vert \vert \hat{w}\vert\vert}y_i(\hat{w}^T\hat{x}_i+b) \geq M \forall i=1,2,\dots, n, 
\]
or just 
\[
y_i(\hat{w}^T\hat{x}_i+b) \geq M\vert \vert \hat{w}\vert\vert \forall i. 
\]
If we scale the equation so that $\vert \vert \hat{w}\vert\vert = 1/M$, we have to find the minimum of 
$\hat{w}^T\hat{w}=\vert \vert \hat{w}\vert\vert$ (the norm) subject to the condition
\[
y_i(\hat{w}^T\hat{x}_i+b) \geq 1 \forall i.
\]

We have thus defined our margin as the invers of the norm of $\hat{w}$. We want to minimize the norm in order to have a as large as possible margin $M$. Before we proceed, we need to remind ourselves about Lagrangian multipliers. 

% !split
\subsection*{A quick reminder on Lagrangian multipliers}

Consider a function of three independent variables $f(x,y,z)$ . For the function $f$ to be an
extreme we have
\[
df=0.
\]
A necessary and sufficient condition is
\[
\frac{\partial f}{\partial x} =\frac{\partial f}{\partial y}=\frac{\partial f}{\partial z}=0,
\]
due to
\[
df = \frac{\partial f}{\partial x}dx+\frac{\partial f}{\partial y}dy+\frac{\partial f}{\partial z}dz.
\]
In many problems the variables $x,y,z$ are often subject to constraints (such as those above for the margin)
so that they are no longer all independent. It is possible at least in principle to use each 
constraint to eliminate one variable
and to proceed with a new and smaller set of independent varables.

The use of so-called Lagrangian  multipliers is an alternative technique  when the elimination
of variables is incovenient or undesirable.  Assume that we have an equation of constraint on 
the variables $x,y,z$
\[
\phi(x,y,z) = 0,
\]
 resulting in
\[
d\phi = \frac{\partial \phi}{\partial x}dx+\frac{\partial \phi}{\partial y}dy+\frac{\partial \phi}{\partial z}dz =0.
\]
Now we cannot set anymore
\[
\frac{\partial f}{\partial x} =\frac{\partial f}{\partial y}=\frac{\partial f}{\partial z}=0,
\]
if $df=0$ is wanted
because there are now only two independent variables!  Assume $x$ and $y$ are the independent 
variables.
Then $dz$ is no longer arbitrary.

% !split
\subsection*{Adding the muliplier}

However, we can add to
\[
df = \frac{\partial f}{\partial x}dx+\frac{\partial f}{\partial y}dy+\frac{\partial f}{\partial z}dz,
\]
a multiplum of $d\phi$, viz. $\lambda d\phi$, resulting  in
\[
df+\lambda d\phi = (\frac{\partial f}{\partial z}+\lambda
\frac{\partial \phi}{\partial x})dx+(\frac{\partial f}{\partial y}+\lambda\frac{\partial \phi}{\partial y})dy+
(\frac{\partial f}{\partial z}+\lambda\frac{\partial \phi}{\partial z})dz =0.
\]
Our multiplier is chosen so that
\[
\frac{\partial f}{\partial z}+\lambda\frac{\partial \phi}{\partial z} =0.
\]

We need to remember that we took $dx$ and $dy$ to be arbitrary and thus we must have
\[
\frac{\partial f}{\partial x}+\lambda\frac{\partial \phi}{\partial x} =0,
\]
and
\[
\frac{\partial f}{\partial y}+\lambda\frac{\partial \phi}{\partial y} =0.
\]
When all these equations are satisfied, $df=0$.  We have four unknowns, $x,y,z$ and
$\lambda$. Actually we want only $x,y,z$, $\lambda$ needs not to be determined, 
it is therefore often called
Lagrange's undetermined multiplier.
If we have a set of constraints $\phi_k$ we have the equations
\[
\frac{\partial f}{\partial x_i}+\sum_k\lambda_k\frac{\partial \phi_k}{\partial x_i} =0.
\]

% !split
\subsection*{Setting up the problem}
In order to solve the above problem, we define the following Lagrangian function to be minimized 
\[
{\cal L}(\lambda,b,\hat{w})=\frac{1}{2}\hat{w}^T\hat{w}-\sum_{i=1}^n\lambda_i\left[y_i(\hat{w}^T\hat{x}_i+b)-1\right],
\]
where $\lambda_i$ is a so-called Lagrange multiplier subject to the condition $\lambda_i \geq 0$.

Taking the derivatives  with respect to $b$ and $\hat{w}$ we obtain 
\[
\frac{\partial {\cal L}}{\partial b} = -\sum_{i} \lambda_iy_i=0,
\]
and 
\[
\frac{\partial {\cal L}}{\partial \hat{w}} = 0 = \hat{w}-\sum_{i} \lambda_iy_i\hat{x}_i.
\]
Inserting these constraints into the equation for ${\cal L}$ we obtain
\[
{\cal L}=\sum_i\lambda_i-\frac{1}{2}\sum_{ij}^n\lambda_i\lambda_jy_iy_j\hat{x}_i^T\hat{x}_j,
\]
subject to the constraints $\lambda_i\geq 0$ and $\sum_i\lambda_iy_i=0$. 
We must in addition satisfy the \href{{https://en.wikipedia.org/wiki/Karush%E2%80%93Kuhn%E2%80%93Tucker_conditions}}{Karush-Kuhn-Tucker} (KKT) condition
\[
\lambda_i\left[y_i(\hat{w}^T\hat{x}_i+b) -1\right] \forall i.
\]
\begin{enumerate}
\item If $\lambda_i > 0$, then $y_i(\hat{w}^T\hat{x}_i+b)=1$ and we say that $x_i$ is on the boundary.

\item If $y_i(\hat{w}^T\hat{x}_i+b)> 1$, we say $x_i$ is not on the boundary and we set $\lambda_i=0$. 
\end{enumerate}

\noindent
When $\lambda_i > 0$, the vectors $\hat{x}_i$ are called support vectors. They are the vectors closest to the line (or hyperplane) and define the margin $M$. 

% !split
\subsection*{The problem to solve}

We can rewrite 
\[
{\cal L}=\sum_i\lambda_i-\frac{1}{2}\sum_{ij}^n\lambda_i\lambda_jy_iy_j\hat{x}_i^T\hat{x}_j,
\]
and its constraints in terms of a matrix-vector problem where we minimize w.r.t. $\lambda$ the following problem
\[
\frac{1}{2} \hat{\lambda}^T\begin{bmatrix} y_1y_1\hat{x}_1^T\hat{x}_1 & y_1y_2\hat{x}_1^T\hat{x}_2 & \dots & \dots & y_1y_n\hat{x}_1^T\hat{x}_n \\
y_2y_1\hat{x}_2^T\hat{x}_1 & y_2y_2\hat{x}_2^T\hat{x}_2 & \dots & \dots & y_1y_n\hat{x}_2^T\hat{x}_n \\
\dots & \dots & \dots & \dots & \dots \\
\dots & \dots & \dots & \dots & \dots \\
y_ny_1\hat{x}_n^T\hat{x}_1 & y_ny_2\hat{x}_n^T\hat{x}_2 & \dots & \dots & y_ny_n\hat{x}_n^T\hat{x}_n \\
\end{bmatrix}\hat{\lambda}-\mathbb{1}\hat{\lambda}, 
\]
subject to $\hat{y}^T\hat{\lambda}=0$. 

% !split
\subsection*{Examples with kernels}

\begin{minted}[fontsize=\fontsize{9pt}{9pt},linenos=false,mathescape,baselinestretch=1.0,fontfamily=tt,xleftmargin=7mm]{python}
from IPython.display import set_matplotlib_formats, display
import pandas as pd
import numpy as np
import matplotlib.pyplot as plt
import mglearn
from cycler import cycler
from sklearn.linear_model import LogisticRegression
from sklearn.svm import LinearSVC
from sklearn.datasets import make_blobs


X, y = make_blobs(centers=4, random_state=8)
y = y % 2

mglearn.discrete_scatter(X[:, 0], X[:, 1], y)
plt.xlabel("Feature 0")
plt.ylabel("Feature 1")
plt.show()

from sklearn.svm import LinearSVC
linear_svm = LinearSVC().fit(X, y)

mglearn.plots.plot_2d_separator(linear_svm, X)
mglearn.discrete_scatter(X[:, 0], X[:, 1], y)
plt.xlabel("Feature 0")
plt.ylabel("Feature 1")
\end{minted}

\begin{minted}[fontsize=\fontsize{9pt}{9pt},linenos=false,mathescape,baselinestretch=1.0,fontfamily=tt,xleftmargin=7mm]{python}
# add the squared first feature
X_new = np.hstack([X, X[:, 1:] ** 2])


from mpl_toolkits.mplot3d import Axes3D, axes3d
figure = plt.figure()
# visualize in 3D
ax = Axes3D(figure, elev=-152, azim=-26)
# plot first all the points with y==0, then all with y == 1
mask = y == 0
ax.scatter(X_new[mask, 0], X_new[mask, 1], X_new[mask, 2], c='b',
           cmap=mglearn.cm2, s=60, edgecolor='k')
ax.scatter(X_new[~mask, 0], X_new[~mask, 1], X_new[~mask, 2], c='r', marker='^',
           cmap=mglearn.cm2, s=60, edgecolor='k')
ax.set_xlabel("feature0")
ax.set_ylabel("feature1")
ax.set_zlabel("feature1 ** 2")
\end{minted}

\begin{minted}[fontsize=\fontsize{9pt}{9pt},linenos=false,mathescape,baselinestretch=1.0,fontfamily=tt,xleftmargin=7mm]{python}
linear_svm_3d = LinearSVC().fit(X_new, y)
coef, intercept = linear_svm_3d.coef_.ravel(), linear_svm_3d.intercept_

# show linear decision boundary
figure = plt.figure()
ax = Axes3D(figure, elev=-152, azim=-26)
xx = np.linspace(X_new[:, 0].min() - 2, X_new[:, 0].max() + 2, 50)
yy = np.linspace(X_new[:, 1].min() - 2, X_new[:, 1].max() + 2, 50)

XX, YY = np.meshgrid(xx, yy)
ZZ = (coef[0] * XX + coef[1] * YY + intercept) / -coef[2]
ax.plot_surface(XX, YY, ZZ, rstride=8, cstride=8, alpha=0.3)
ax.scatter(X_new[mask, 0], X_new[mask, 1], X_new[mask, 2], c='b',
           cmap=mglearn.cm2, s=60, edgecolor='k')
ax.scatter(X_new[~mask, 0], X_new[~mask, 1], X_new[~mask, 2], c='r', marker='^',
           cmap=mglearn.cm2, s=60, edgecolor='k')

ax.set_xlabel("feature0")
ax.set_ylabel("feature1")
ax.set_zlabel("feature1 ** 2")

ZZ = YY ** 2
dec = linear_svm_3d.decision_function(np.c_[XX.ravel(), YY.ravel(), ZZ.ravel()])
plt.contourf(XX, YY, dec.reshape(XX.shape), levels=[dec.min(), 0, dec.max()],
             cmap=mglearn.cm2, alpha=0.5)
mglearn.discrete_scatter(X[:, 0], X[:, 1], y)
plt.xlabel("Feature 0")
plt.ylabel("Feature 1")
\end{minted}

\begin{minted}[fontsize=\fontsize{9pt}{9pt},linenos=false,mathescape,baselinestretch=1.0,fontfamily=tt,xleftmargin=7mm]{python}
from sklearn.svm import SVC
X, y = mglearn.tools.make_handcrafted_dataset()                                                                  
svm = SVC(kernel='rbf', C=10, gamma=0.1).fit(X, y)
mglearn.plots.plot_2d_separator(svm, X, eps=.5)
mglearn.discrete_scatter(X[:, 0], X[:, 1], y)
# plot support vectors
sv = svm.support_vectors_
# class labels of support vectors are given by the sign of the dual coefficients
sv_labels = svm.dual_coef_.ravel() > 0
mglearn.discrete_scatter(sv[:, 0], sv[:, 1], sv_labels, s=15, markeredgewidth=3)
plt.xlabel("Feature 0")
plt.ylabel("Feature 1")

fig, axes = plt.subplots(3, 3, figsize=(15, 10))

for ax, C in zip(axes, [-1, 0, 3]):
    for a, gamma in zip(ax, range(-1, 2)):
        mglearn.plots.plot_svm(log_C=C, log_gamma=gamma, ax=a)
        
axes[0, 0].legend(["class 0", "class 1", "sv class 0", "sv class 1"],
                  ncol=4, loc=(.9, 1.2))

\end{minted}

% ------------------- end of main content ---------------

\end{document}

